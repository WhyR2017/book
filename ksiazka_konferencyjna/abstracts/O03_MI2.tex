\providecommand{\main}{..} 
\documentclass[\main/boa.tex]{subfiles}

\begin{document}

	\begin{minipage}[t]{0.915\textwidth}
		\center     
		\includegraphics[width=100px]{img/logos.bw/mi2_data_lab_napis.png} 
	\end{minipage}

\Large \textbf {Grupa MI$^{2}$}


\vskip 0.3cm
\normalsize 
MI$^{2}$ (czytaj: mi kwadrat) powstała jako pomost pomiędzy MIM UW i MiNI PW, czyli dwoma wydziałami pełnymi pasjonatów analizy danych oraz tworzenia narzędzi matematycznych i informatycznych do analizy danych. Zaczęło się od UW i PW ale w grupie działały też osoby z innych uczelni, miast czy nawet krajów zarówno studenci jak i absolwenci.

Działamy w modelu think \& do. Analiza danych nie jest wartością samą w sobie. Ważne by była odpowiedzią na rzeczywiste potrzeby i prowadziła do lepszych decyzji. Rozwijamy umiejętności identyfikacji problemu, rozwiązywania problemu i wdrażania rozwiązania w oparciu o dane i zaawansowane metody analityczne, skalowane narzędzia informatyczne. Robimy to realizując projekty przy których nie tylko można się czegoś dowiedzieć, ale też można coś użytecznego zrobić a wyniki analiz przełożyć na faktyczne akcje.

\end{document}
