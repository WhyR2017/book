\providecommand{\main}{..} 
\documentclass[\main/boa.tex]{subfiles}

\begin{document}

\section{Interaktywne wizualizacje w R i plotly - case study}
\begin{minipage}[t]{0.915\textwidth}
	\center     
    \includegraphics[width=60px]{img/workshops/czarno_biale/pocalewicz-crop.png} 
\end{minipage}

\begin{minipage}{0.915\textwidth}
\centering
{\bf \index[a]{Ocalewicz Piotr} Piotr Ocalewicz}
\end{minipage}

\vskip 0.3cm

\begin{affiliations}
\begin{minipage}{0.915\textwidth}
\centering
\large Ocado Technology  \\[2pt]
\end{minipage}
\end{affiliations}

\vskip 0.8cm

\opiswarsztatu Celem warsztatu jest zapoznanie użytkowników z możliwościami tworzenia interaktywnych wizualizacji danych korzystając z połączenia środowisk R, plotly oraz leaflet. To świetna okazja, żeby rozszerzyć swoje umiejętności w zakresie wizualizacji danych i nauczyć się tworzyć ciekawe i niebanalne podsumowania swojej pracy.

Warsztat prowadzony będzie w formie 'case study' - przejdziemy krok po kroku przez kolejne kroki analizy od krótkiego zapoznania się z danymi, poprzez stworzenie różnych interaktywnych wizualizacji aż po rozwiązanie problemu, który przed sobą postawiliśmy.

W trakcie warsztatu stworzymy kompletny dokument w formacie html zawierający podsumowanie analizowanych danych oraz stworzone przez nas grafiki. Omówimy zarówno podstawowe rodzaje wykresów, te bardziej zaawansowane jak również sposoby ich połączenia w jednym, estetycznym podsumowaniu.

W trakcie szkolenia każdy uczestnik otrzyma wydrukowane 'ściągawki' zawierające najważniejsze funkcje i składnię omawianych pakietów.

\planwarsztatu
\begin{enumerate}
\item Omówienie zbioru danych i problemu do rozwiązania
\item Krótkie wprowadzenie do pakietów ggplot2 oraz rmarkdown
\item Środowisko plotly i jego współpraca z R
\item Podstawowe typy wykresów
\item Zaawansowane wykresy i sposoby ich edycji
\item Dynamiczne zmienianie zawartości wykresów - guziki, suwaki itd.
\item Interaktywna wizualizacja na mapach
\item Podsumowanie warsztatu i wyników analizy
\end{enumerate}	 

\end{document}
