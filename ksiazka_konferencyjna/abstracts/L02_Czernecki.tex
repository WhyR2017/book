\providecommand{\main}{..} 
\documentclass[\main/boa.tex]{subfiles}

\begin{document}

\section{FasteR - kiedy warto przenieść obliczenia na superkomputery?}

\begin{minipage}{0.915\textwidth}
	\centering
  {\bf \index[a]{Czernecki Bartosz} Bartosz Czernecki}
\end{minipage}



\begin{affiliations}
\begin{minipage}{0.915\textwidth}
\centering
Uniwersytet im. A. Mickiewicza w Poznaniu  \\[-2pt]
\end{minipage}
\end{affiliations}

\vskip 0.3cm

W prezentacji przedstawiono przykładowe rozwiązania (i ograniczenia) związane z przyspieszaniem kodu R na przykładzie danych meteorologicznych. Omówiono korzyści płynące z unikania pętli, wektoryzacji obliczeń, wykorzystania bardziej wydajnych pakietów a także przepisania kodu R do C(++)/Fortrana i konsekwencji wynikających ze zrównoleglania obliczeń. 
Przedstawiono także kilka własnych doświadczeń dotyczących przenoszenia obliczeń na superkomputery (HPC) znajdujące się w Poznańskim i Wrocławskim Centrum Superkomputerowo-Sieciowym w ramach dostępnych dla zastosowań naukowych (i komercyjnych) grantów obliczeniowych.



\end{document}
