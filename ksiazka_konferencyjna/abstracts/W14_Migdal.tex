\providecommand{\main}{..} 
\documentclass[\main/boa.tex]{subfiles}

\begin{document}

\section{mirt: skalowanie odpowiedzi lepsze niż PCA}
\begin{minipage}[t]{0.915\textwidth}
	\center     
    \includegraphics[width=60px]{img/workshops/czarno_biale/pmigdal_crop.png} 
\end{minipage}

\begin{minipage}{0.915\textwidth}
\centering
{\bf \index[a]{Migdał Piotr} Piotr Migdał}
\end{minipage}

\vskip 0.3cm

\begin{affiliations}
\begin{minipage}{0.915\textwidth}
\centering
\large deepsense.io, freelancer  \\[2pt]
\end{minipage}
\end{affiliations}

\vskip 0.8cm

\opiswarsztatu Item Response Theory jest modelem analizy danych, w której szukamy zmiennej ukrytej wyjaśniającej dane. Np. zamieniamy wiele odpowiedzi z ankiety na jedną zmienną odpowiadającą zadowoleniu klienta, czy też estymujemy pewną cechę charakteru na podstawie kwestionariusza. Innym zastosowaniem jest skalowanie wyników egzaminów w sposób mądrzejszy niż liczenie sumy punktów (nie każde zadanie jest równie trudne, niektóre zadania mogą być losowe).

Typowe sposoby (np. liczenie pierwszej składowej w PCA) nie uwzględniają nieliniowości zmiennych.

Pakiet mirt (Multidimensional Item Response Theory) jest wydajnym i wszechstronnym pakietem do praktycznych zastosowań IRT.

\planwarsztatu
\begin{enumerate}
\item wprowadzenie do Item Response Theory
\item różne modele zmiennych odpowiedzi (też: gradualne)
\item szukanie zmiennej ukrytej
\item generowanie sztucznych odpowiedzi
\item ćwiczenie praktyczne: analiza danych maturalnych
\end{enumerate}	 

\end{document}