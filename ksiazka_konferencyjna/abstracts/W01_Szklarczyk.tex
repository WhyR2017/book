\providecommand{\main}{..} 
\documentclass[\main/boa.tex]{subfiles}

\begin{document}

\section{Analiza danych sondażowych w R}

\begin{minipage}{0.915\textwidth}
\centering
{\bf \index[a]{Szklarczyk Dariusz} Dariusz Szklarczyk, \index[a]{Otręba-Szklarczyk Agnieszka} Agnieszka Otręba-Szklarczyk}
\end{minipage}

\vskip 0.3cm

\begin{affiliations}
\begin{minipage}{0.915\textwidth}
\centering
\large Fundacja Rozwoju Badań Społecznych  \\[2pt]
\end{minipage}
\end{affiliations}

\vskip 0.8cm

\opiswarsztatu Wśród badaczy społecznych korzystających używających R do pracy z danymi panuje dość powszechna zgoda, że dużo łatwiej jest w R zbudować np. model regresji liniowej, niż przygotować dane do analizy i wykonać najprostsze, a najpowszechniej stosowane w praktyce analizy, np. rozkłady procentowe dla wielokrotnych odpowiedzi czy tabele krzyżowe. Tymczasem, ze względu na dużą liczbę zmiennych kategorialnych stosowanych w badaniach społecznych, zwłaszcza sondażowych, umiejętność ta jest niezbędna zarówno na etapie wstępnej eksploracji danych, jak i prezentowania prostych zależności. Jednocześnie trudności w przeprowadzeniu prostych, a niezbędnych operacji, skutecznie zniechęcają (niesłusznie!) do nauki R osoby, które przyzwyczaiły się do prostego i błyskawicznego ich sporządzaniu przy pomocy komercyjnych pakietów, takich jak SPSS czy Statistica. Celem warsztatu jest zaprezentowanie najbardziej przydatnych funkcji i pakietów służących przygotowaniu danych z badań sondażowych do analizy (m.in. pakiet dplyr) i analizie danych sondażowych, ze szczególnym uwzględnieniem danych kategorialnych i zależności między nimi (m.in. rozkłady procentowe, wielokrotne odpowiedzi, tabele krzyżowe). Nacisk zostanie również położony na przygotowanie planu analizy. Aby zmaksymalizować użyteczność R w kontekście badań sondażowych, zaprezentowane zostaną również pakiety i funkcje służące do doboru próby (m.in. sampling) oraz ważenia danych sondażowych (weights). Warsztat adresowany jest w szczególności dla badaczy społecznych, politologów, socjologów, badaczy rynku i innych osób, które w pracy zawodowej analizują dane sondażowe i chcieliby zacząć robić to w R, z naciskiem na analizę refleksyjną, niezautomatyzowaną.

\planwarsztatu
\begin{enumerate}
\item Przygotowanie danych do analizy – czyszczenie bazy danych, rekodowanie danych.
\item Eksploracja danych sondażowych – przygotowanie planu analizy, zestawy wielokrotnych odpowiedzi, statystyki opisowe, wizualizacja danych.
\item Analiza zależności między danymi sondażowymi – tabele krzyżowe, analiza korespondencji, analiza zależności między zmiennymi ilościowymi, tworzenie indeksów.
\item Dobór próby losowej (prostej, warstwowanej) i ważenie próby (klasyczne oraz rake weights).
\end{enumerate}	 

\pakiety dplyr, sampling, weights, questionr, ca, ggplot, gmodels

\umiejetnosci Zapraszamy wszystkie osoby zainteresowane tematem analizy danych sondażowych, społecznych. Mile widziane podstawowe doświadczenie w prowadzeniu badań sondażowych, ankiet itp.

\wymagania Zainstalowanie R, RStudio. Dane zostaną przekazane w trakcie warsztatu.

\end{document}
