\providecommand{\main}{..} 
\documentclass[\main/boa.tex]{subfiles}

\begin{document}

\section{Zastosowanie R w Power BI}
\begin{minipage}[t]{0.915\textwidth}
	\center     
    \includegraphics[width=60px]{img/workshops/czarno_biale/ddetko-crop.png} 
\end{minipage}

\begin{minipage}{0.915\textwidth}
\centering
{\bf \index[a]{Detko Dawid} Dawid Detko}
\end{minipage}

\vskip 0.3cm

\begin{affiliations}
\begin{minipage}{0.915\textwidth}
\centering
\large Predica  \\[2pt]
\end{minipage}
\end{affiliations}

\vskip 0.8cm

\opiswarsztatu Bardzo często potrzebujemy narzędzia, z którego ma korzystać ktoś co nie zna języków skryptowych, nie używa R Studio, czy notebooków. Możliwosć taką daje obecnie najbardziej popularny produkt na świecie to tzw Self-BI, czyli PowerBI firmy Microsoft. Produkt ten jest w pełni darmowy i daje możliwość zarówno pobierania danych ze skryptów w języku R, jak i tworzyć wizualizacje przy użyciu tego języka.

\planwarsztatu
\begin{enumerate}
\item Przedstawienie PowerBI (w krótki i przystępny sposób)
\item Język R źródłem danych
\item Łączenie źródeł danych z języka R i innych źródeł
\item Wizualizacje w języku R
\end{enumerate}	 

\end{document}
