\providecommand{\main}{..} 
\documentclass[\main/boa.tex]{subfiles}

\begin{document}

\section{Efektywna i efektowna wizualizacja w ggplot2}

\begin{minipage}{0.915\textwidth}
\centering
{\bf \index[a]{Ćwiakowski Piotr} Piotr Ćwiakowski}
\end{minipage}

\vskip 0.3cm

\begin{affiliations}
\begin{minipage}{0.915\textwidth}
\centering
\large Uniwersytet Warszawski  \\[2pt]
\end{minipage}
\end{affiliations}

\vskip 0.8cm

\opiswarsztatu Przedstawienie zaawansowanych funkcji i rozszerzeń pakietu ggplot2. Po warsztacie użytkownik zna zaawansowane możliwości pakietu ggplot2 (m. in. interaktywne wykresy) oraz poznał zasady poprawnej wizualizacji danych

\planwarsztatu
\begin{enumerate}
\item Wprowadzenie do tidyverse, grammar of graphics i ggplot2
\item Zasady działania ggplot2
\item Przegląd geometrii w ggplot2 (z szczególnym uwzględnieniem zaawansowanych i nietypowych
\item Przegląd rozszerzeń do ggplot2
\item Sztuka tworzenia wykresów
\item Tworzenie złożonych i zaawansowanych wykresów w ggplot2 w praktyce
\end{enumerate}	 

\pakiety tidyverse, ggplot2 extensions

\umiejetnosci Podstawowy R, mile widziane doświadczenie w analizie danych (niekoniecznie w R)

\wymagania Zainstalowanie R (z opcjonalną, ale rekomendowaną nakładką R Studio), zainstalowanie pakietu tidyverse i wybranych pakietów z rodziny ggplot2 extensions

\end{document}
