\providecommand{\main}{..} 
\documentclass[\main/boa.tex]{subfiles}

\begin{document}

\subsection{Planowanie pojemności i wydajności w celu przeciwdziałania awariom}

\begin{minipage}{0.915\textwidth}
	\centering
  {\bf \index[a]{Bigos Robert} Robert Bigos}
\end{minipage}


\begin{affiliations}
\begin{minipage}{0.915\textwidth}
\centering
Wcześniej pracował dla fim takich jak IBM, SAP, Sabre. Obecnie doradza klientom ciesząc się życiem :)  \\[-2pt]
\end{minipage}
\end{affiliations}

\vskip 0.3cm

 Nie ma zasobów nieskończonych, nie ma zasobów idealnych. Każdy system komputerowy to zbiór bardzo wielu zależnych od siebie kolejek które mogą zostać przeciążone i wpływać na pozostałe. Przy odpowiednim przeciążeniu wszystko kiedyś ulegnie awarii. Coraz częściej spotykamy się ze zjawiskiem “capacity rolling disaster” czyli awariami w których pierwotna przyczyna tylko inicjuje łańcuch zdarzeń. Duże systemy instrumentacyjne (logi/metryki) zbierają rocznie TBy danych, w rozdzielczość na poziomie (ms)sekund. Jak w tym wszystkim doszukać się wzorców i pierwotnych przyczyn awarii by im przeciwdziałać? Jak zaplanować zmiany aby zapewnić odpowiednią pojemność systemu w czasie?
 
 Do zabawy z danymi wykorzystamy R i wizualizacje grafów animowanych po czasie.
 
\end{document}
