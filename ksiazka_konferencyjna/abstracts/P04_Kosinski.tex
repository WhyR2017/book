\providecommand{\main}{..} 
\documentclass[\main/boa.tex]{subfiles}

\begin{document}

\subsection{Visualizing statistical analysis methods on the data from The Cancer Genome Atlas Study available in the RTCGA Family of R packages}

\begin{minipage}{0.915\textwidth}
\centering
{\bf \index[a]{Kosiński Marcin}  Marcin Kosiński, \index[a]{Chodor Witold}  Witold Chodor, \index[a]{Biecek Przemysław}  Przemysław Biecek}
\end{minipage}


\begin{affiliations}
\begin{minipage}{0.915\textwidth}
\centering
MI$^{2}$   \\[-2pt]
\end{minipage}
\end{affiliations}

\vskip 0.3cm

The following poster presents RTCGA: a family of R packages with data from The Cancer Genome Atlas Project (TCGA) study. TCGA is a comprehensive and coordinated effort to accelerate our understanding of the
molecular basis of cancer through the application of genome analysis technologies, including large-scale genome sequencinga. 
We converted selected datasets from this study into few separate packages that are hosted on Bioconductor.
These R packages make selected datasets easier to access and manage. Data sets in RTCGA.data packages are large and cover complex
relations between clinical outcomes and genetic background.

These packages will be useful for at least three audiences: biostatisticians that work with cancer data; researchers that are working on large scale algorithms, for them RTGCA data will be a perfect blasting site; teachers that are presenting data analysis method on real data problems. In this poster we present applications of the statistical modeling methods to the cancer’s subtypes basing on the available cancer data of over 8,000 patients.

\end{document}
