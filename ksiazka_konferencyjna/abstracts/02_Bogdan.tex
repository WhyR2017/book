\providecommand{\main}{..} 
\documentclass[\main/boa.tex]{subfiles}

\begin{document}

\section{Sorted L-One Penalized Estimation}


\begin{minipage}{0.915\textwidth}
	\centering
  {\bf \LARGE \index[a]{Bogdan Małgorzata} Małgorzata Bogdan}
\end{minipage}




\begin{affiliations}
\begin{minipage}{0.915\textwidth}
\centering
\large Uniwersytet Wrocławski  \\[1pt]
Kontakt: \href{mailto:Malgorzata.Bogdan@uwr.edu.pl}{\nolinkurl{Malgorzata.Bogdan@uwr.edu.pl}}\\
\end{minipage}
\end{affiliations}


SLOPE (Sorted L-One Penalized Estimation) to nowy algorytm optymalizacji wypuklej, sluzacy do redukcji wymiaru w duzych bazach danych. W czasie wykladu zaprezentujemy zastosowanie SLOPE do szeregu problemow statystycznych, jak np. wybor istotnych zmiennych w regresji liniowej i logistycznej czy optymalizacja portfela, a takze omowimy odpowiednie pakiety w R. 

\bio
Małgorzata Bogdan uzyskała dyplom magistra z matematyki (1992) i doktora nauk matematycznych (1996, specjalność statystyka matematyczna) na Politechnice Wrocławskiej i habilitację z nauk technicznych (2009, specjalność informatyka) w Instytucie Podstaw Informatyki Polskiej Akademii Nauk.

Pracuje jako profesor nadzwyczajny w Instytucie Matematyki Uniwersytetu Wrocławskiego. Głównym obszarem jej zainteresowań naukowych jest redukcja wymiaru w dużych zbiorach danych i zastosowania do analizy danych genetycznych.

Jest współautorką wraz z Florianem Frommletem i Davidem Ramseyem, książki Phenotypes and Genotypes. Search for influential genes.


\end{document}
