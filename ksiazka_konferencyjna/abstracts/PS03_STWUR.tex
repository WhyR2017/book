\providecommand{\main}{..} 
\documentclass[\main/boa.tex]{subfiles}

\begin{document}
	
	\begin{minipage}[t]{0.915\textwidth}
		\center     
		\includegraphics[width=100px]{img/logos.bw/stwur.png} 
	\end{minipage}
	
	\Large \textbf {STWUR}
	
	
	\vskip 0.3cm
	\normalsize 
	Stowarzyszenie Wrocławskich Użytkowników R (STWUR) to dolnośląska grupa miłośników R. Na spotkaniach stawiamy sobie trzy cele. Pierwszy to integracja eRowego środowiska, ludzi, którzy w R programują lub są nim zainteresowani. Drugi to wspólna analiza inspirujących zbiorów danych, wyciąganie wniosków i komunikowanie ich poza grupę. Last but not least, spotkania STWURa to idealne miejsce na wymianę doświadczeń, naukę nowych pakietów i szansa na (lepsze) poznanie R.
	
	W przeciwieństwie do większości spotkań użytkowników R, STWUR jest zaplanowany jako cykl spotkań, których wspólnym motywem jest jeden zbiór danych. Zaczynamy od danych z Diagnozy Społecznej, które mamy zamiar ze spotkania na spotkanie coraz głębiej eksplorować i poznawać.
	
\end{document}
