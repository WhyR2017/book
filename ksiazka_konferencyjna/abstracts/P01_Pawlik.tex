\providecommand{\main}{..} 
\documentclass[\main/boa.tex]{subfiles}

\begin{document}

\subsection{Analizy gleboznawcze w R}

\begin{minipage}{0.915\textwidth}
	\centering
  {\bf \index[a]{Pawlik Łukasz} Łukasz Pawlik, \index[a]{Samonil Pavel} Pavel Samonil}
\end{minipage}

\begin{affiliations}
\begin{minipage}{0.915\textwidth}
\centering
Uniwersytet Pedagogiczny, Kraków  \\[-2pt]
Instytut Ekologii Lasu, Brno, Czechy  \\[-2pt]
\end{minipage}
\end{affiliations}

\vskip 0.3cm

 Wyniki laboratoryjnych analiz chemicznych i fizycznych próbek gleb pobranych \break z poligonów badawczych w Gorcach, rezerwacie Zofin (Czechy) i stanie Michigan (USA) zostały poddane analizie i wizualizacji w pakiecie statystycznym R. W tym celu wykorzystano następujące pakiety: stats, aqp, corrplot, FSA, ggplot2, vegan, psych. 

\end{document}
