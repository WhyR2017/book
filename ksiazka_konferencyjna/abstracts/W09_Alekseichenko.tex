\providecommand{\main}{..} 
\documentclass[\main/boa.tex]{subfiles}

\begin{document}

\section{XGBoost rządzi}
\begin{minipage}[t]{0.915\textwidth}
	\center     
    \includegraphics[width=60px]{img/workshops/czarno_biale/vladimir-crop.png} 
\end{minipage}

\begin{minipage}{0.915\textwidth}
\centering
{\bf \index[a]{Alekseichenko Vladimir} Vladimir Alekseichenko}
\end{minipage}

\vskip 0.3cm

\begin{affiliations}
\begin{minipage}{0.915\textwidth}
\centering
\large General Electric \\[2pt]
\end{minipage}
\end{affiliations}

\vskip 0.8cm

\opiswarsztatu XGBoost to jest jedna najlepszych implementacji "Gradient Boosting" z punktu widzenia praktycznego.9
Dlaczego warto?
\begin{enumerate}
	\item Wynik (czyli zwykle jeden z najlepszych)
	\item Czas na naukę i predykcję (potrafi używać wszystkie dostępne rdzenie)
	\item Odporność na przeuczenia się (poprzez różne parametry regularyzacji)
	\item Stabilność (można spokojnie używać na produkcje)
\end{enumerate}

\planwarsztatu
\begin{enumerate}
\item Zrozumienie biznes problemu
\item Zrozumienie danych
\item Budowa bardzo prostego modelu (base-line)
\item Przypomnienie co to jest drzewa decezyjne
\item Uruchomienie prostego modelu xgboost
\item Generowanie cech (feature engineering)
\item Budowanie bardziej zaawansowanego modelu
\item Optymalizacja hyperparametrów
\item Inne (zaawansowane) triki (opcjonalnie)
\end{enumerate}	 

\pakiety xgboost, data.table, e1071, caret, rBayesianOptimization

\umiejetnosci Warsztat może być ciekawy dla osób które dopiero zaczynają, jak i dla średnio-zaawansowanych (z mojej wiedzy mało osób kojarzy i tym bardziej używa XGBoost w praktyce, chociaż to zmienia się bardzo szybko w czasie).
Natomiast warto rozumieć podstawy:
\begin{enumerate}
	\item uczenie maszynowe (machine learning)
	\item cechy (features)
	\item model, np. liniowy
	\item przeuczenie się (overfitting)
	\item walidacja (model evaluation)
\end{enumerate}
Fajnie będzie jeżeli sprawdzisz (przypomnisz) jak działają drzewa decezyjne (decision trees).

\wymagania 
\begin{enumerate}
\item Mieć laptop z potrzebnymi pakietami (przede wszystkim xgboost)
\item Pobrać dane z Kaggle
\item Pomyśleć nad problemem przed warsztatem (może nawet spróbować go rozwiązać w najlepszy możliwy sposób - użyć dowolny model, który się zna)
\end{enumerate}

\end{document}
