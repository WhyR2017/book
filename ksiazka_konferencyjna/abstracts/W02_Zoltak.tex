\providecommand{\main}{..} 
\documentclass[\main/boa.tex]{subfiles}

\begin{document}

\section{Złożone schematy doboru próby - pakiet survey}
\begin{minipage}[t]{0.915\textwidth}
	\center     
    \includegraphics[width=60px]{img/workshops/czarno_biale/zoltak.png} 
\end{minipage}

\begin{minipage}{0.915\textwidth}
\centering
{\bf \index[a]{Żółtak Tomasz} Tomasz Żółtak}
\end{minipage}

\vskip 0.3cm

\begin{affiliations}
\begin{minipage}{0.915\textwidth}
\centering
\large Instytut Badań Edukacyjnych  \\[2pt]
\end{minipage}
\end{affiliations}

\vskip 0.8cm

\opiswarsztatu Duża część dostępnych powszechnie danych z badań sondażowych pochodzi z projektów, w~których wykorzystywane są złożone schematy doboru prób badawczych. W szczególności dotyczy to międzynarodowych badań porównawczych w dziedzinie edukacji (np. TIMSS, PIRLS, PISA, PIAAC) czy nauk społecznych (np. ESS), ale też badań dotyczących zdrowotności i epidemiologii. Analiza tych danych przy pomocy klasycznie wykorzystywanych technik, zakładających, że próba została dobrana w sposób prosty, może prowadzić do błędnych wniosków, w~szczególności w zakresie wielkości błędów standardowych (a w konsekwencji istotności statystycznych). Zasadniczym celem warsztatu jest zapoznanie uczestników z możliwościami pakietu „survey”, który umożliwia analizę tego rodzaju danych w R z wykorzystaniem technik adekwatnych do prób dobranych w sposób złożony: z wykorzystaniem stratyfikacji, doboru wielostopniowego, czy zespołowego.

\planwarsztatu
\begin{enumerate}
\item Złożone schematy doboru prób badawczych – jak i po co się to robi?
\item Pakiet „survey” - jego możliwości i ograniczenia.
\item Definiowanie typowych złożonych schematów doboru próby w pakiecie „survey”.
\item Estymacja typowych statystyk opisowych.
\item Przerwa.
\item Wizualizacja danych przy pomocy pakietów „survey” i „ggplot2”.
\item Regresja liniowa i uogólniona regresja liniowa.
\item Poststratyfikacja i techniki pokrewne (jeśli ktoś jest zainteresowany, może rzucić okiem na tą prezentację).
\end{enumerate}
W czasie warsztatu wykorzystywane będą dane z Europejskiego Sondażu Społecznego i badań PISA.

\end{document}
