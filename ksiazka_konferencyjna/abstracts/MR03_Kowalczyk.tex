\providecommand{\main}{..} 
\documentclass[\main/boa.tex]{subfiles}

\begin{document}

\subsection{Zastosowanie analizy szeregów czasowych w modelowaniu ryzyka niespłacalności}

\begin{minipage}{0.915\textwidth}
	\centering
  {\bf \index[a]{Kowalczyk Dorota} Dorota Kowalczyk}
\end{minipage}


\begin{affiliations}
\begin{minipage}{0.915\textwidth}
\centering
UBS \\[-2pt]
\end{minipage}
\end{affiliations}

\vskip 0.3cm

 W następstwie kryzysu finansowego 2007 -2009 banki przywiązują dużą wagę do prognozowania strat dla potrzeb testów stresu. Testy stresu polegają na estymowaniu straty w zadanym scenariuszu makroekonomicznym. Elementem prognozowania strat z tytułu ryzyka kredytowego jest zwykle prognozowanie ryzyka niespłacalności (PD - probability of default). Prognozy takie tworzone są z wykorzystaniem makroekonomicznych czy też finansowych szeregów czasowych. Po krótkim wprowadzeniu do modelowaniu ryzyka niespłacalności dla potrzeb testów stresu i do wybranych elementów analizy szeregów czasowych (np stacjonarność czy rząd integracji), skupimy się pakietach zwykle wykorzystywanych do takiej analizy: uroot, urca , tseries. Niektóre z zastosowanych w tych pakietach rozwiązań zostaną poddane krytyce, inne opatrzone komentarzem jak je lepiej stosować.

Pakiety: uroot, urca , tseries

\end{document}
