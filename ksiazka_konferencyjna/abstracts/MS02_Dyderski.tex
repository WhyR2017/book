\providecommand{\main}{..} 
\documentclass[\main/boa.tex]{subfiles}

\begin{document}

\subsection[Relacje między lasem a klimatem w różnych skalach przestrzennych – jak to ugRyźć?]{Relacje między lasem a klimatem w różnych skalach przestrzennych – jak to ugRyźć?}

\begin{minipage}{0.915\textwidth}
	\centering
  {\bf \index[a]{Dyderski Marcin K.} Marcin K. Dyderski \index[a]{Jagodziński Andrzej M.} Andrzej M. Jagodziński}
\end{minipage}


\begin{affiliations}
\begin{minipage}{0.915\textwidth}
\centering
Instytut Dendrologii Polskiej Akademii Nauk w Kórniku  \\[-2pt]
\end{minipage}
\end{affiliations}

\vskip 0.3cm

Lasy poprzez wiązanie dwutlenku węgla z atmosfery regulują jego stężenie, wpływając na klimat. Z drugiej strony, warunki klimatyczne wyznaczają granice występowania poszczególnych gatunków drzew. Celem prezentacji jest pokazanie przykładów zastosowania bibliotek programu R jako narzędzi które pomagają poznać oba procesy w różnych skalach przestrzennych. Wiązanie dwutlenku węgla przez drzewa związane jest z produkcją biomasy. Pakiet dplR pozwala na analizy sekwencji przyrostów rocznych drzew – szeregów czasowych, skorelowanych głównie z warunkami klimatycznymi oraz wiekiem drzewa. Dzięki znajomości zmian średnicy drzewa i allometrycznych zależności pomiędzy średnicą i masą, możemy wykonać predykcję przyrostu biomasy drzew. Znając zawartość węgla w biomasie drzew możemy obliczyć, jak wiele dwutlenku węgla jest pochłaniane przez drzewa. Do obliczenia biomasy w drzewostanach wykorzystuje się wskaźniki przeliczeniowe – tzw. Biomass Conversion and Expansion Factors (BCEF). Dzięki zastosowaniu odpowiednich BCEF oraz danych z inwentaryzacji leśnych można obliczyć zasoby biomasy oraz związanego dwutlenku węgla w lasach Polski. Jednym z rozwiązań ułatwiających obliczenia w tym procesie jest pakiet dplyr, pozwalający na szybką aplikację odpowiednich wskaźników. Czynniki klimatyczne wpływają na występowanie poszczególnych gatunków drzew. W celu określenia zmian zasięgu ich występowania stosuje się modele rozmieszczenia gatunków, w których predyktorami są parametry bioklimatyczne. Po zastosowaniu takiego modelu do projekcji zmian klimatycznych można określić zmiany optimum klimatycznego dla poszczególnych gatunków. Jednym z algorytmów predykcji rozmieszczenia gatunków jest model MaxEnt, zaimplementowany w pakiecie dismo. R jest narzędziem pozwalającym na analizę złożonych danych z zakresu nauk leśnych. Umożliwia to próbę odpowiedzi na najistotniejsze pytania z pogranicza ekologii i hodowli lasu, mające duże znaczenie zarówno naukowe, jak i praktyczne, w obliczu prognozowanych zmian klimatycznych. 

\end{document}
