\providecommand{\main}{..} 
\documentclass[\main/boa.tex]{subfiles}

\begin{document}

	\begin{minipage}[t]{0.915\textwidth}
		\center     
		\includegraphics[width=100px]{img/logos.bw/mini.png} 
	\end{minipage}

\Large \textbf {Wydział Matematyki i Nauk Informacyjnych Politechiniki Warszawskiej}


\vskip 0.3cm
\normalsize 

Wydział Matematyki i Nauk Informacyjnych powstał w 1999 roku w wyniku podzielenia Wydziału Fizyki Technicznej i Matematyki Stosowanej. Początkowo miał siedzibę w Gmachu Głównym Politechniki Warszawskiej, a w 2012 roku przeniósł się do nowo wybudowanego budynku przy ul. Koszykowej 75.

Wydział może pochwalić się wysoką pozycją naukową (kategoria A) oraz wysokim poziomem kształcenia (wyróżnienie na kierunku Matematyka i pozytywna ocena na kierunku Informatyka Państwowej Komisji Akredytacyjnej), zaangażowaniem kadry naukowo-dydaktycznej w działalność publikacyjną oraz prowadzeniem dużych projektów informatycznych, w tym projektów zakończonych wdrożeniami.

Wydział prowadzi studia pierwszego i drugiego stopnia na kierunkach Matematyka, Informatyka i Computer science (studia w języku angielskim) oraz studia niestacjonarne pierwszego stopnia na kierunku Matematyka, a także studia doktoranckie z matematyki. Wydział uruchamia studia doktoranckie z informatyki od października 2015 roku. Obecnie na Wydziale studiuje około 1000 studentów.

\end{document}
