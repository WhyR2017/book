\providecommand{\main}{..} 
\documentclass[\main/boa.tex]{subfiles}

\begin{document}

\subsection{ Modelowanie ryzyka kredytowego z wykorzystaniem panelowej regresji liniowej}

\begin{minipage}{0.915\textwidth}
	\centering
  {\bf \index[a]{Ochotny Stanisław} Stanisław Ochotny}
\end{minipage}


\begin{affiliations}
\begin{minipage}{0.915\textwidth}
\centering
UBS \\[-2pt]
\end{minipage}
\end{affiliations}

\vskip 0.3cm

Typowe modele ryzyka kredytowego (PD) jako zmienną zależną biorą zmienną binarną, wskazującą, czy w danym okresie klient zdołał spełnić swoje zobowiązania finansowe wynikające z umowy kredytowej. Problem pojawia się, gdy w portfelu kredytów lub segmencie klientów historycznie nie mamy dostatecznie wielu obserwacji wskazujących na niewypełnianie zobowiązań, by zbudować dla niego dobry model statystyczny. Wykład przedstawi inne możliwe zmienne zależne i metody ich modelowania z użyciem panelowej regresji liniowej (pakiet plm).

\end{document}
