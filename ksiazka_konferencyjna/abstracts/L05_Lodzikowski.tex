\providecommand{\main}{..} 
\documentclass[\main/boa.tex]{subfiles}

\begin{document}

\subsection{What drumming taught me about leading a data science team}

\begin{minipage}{0.915\textwidth}
	\centering
  {\bf \index[a]{Lodzikowski Kacper} Kacper Lodzikowski}
\end{minipage}



\begin{affiliations}
\begin{minipage}{0.915\textwidth}
\centering
Pearson  \\[-2pt]
\end{minipage}
\end{affiliations}

\vskip 0.3cm

 This talk provides practical tips on how to lead a data science team by drawing an analogy between the role of a drummer in a band and a team leader. While drummers don’t create the main value of a song (the melody) and they rarely ‘lead’ bands the way lead singers or guitarists do, they are always their band’s backbone because they set and keep time for others to follow. Similarly, good data science leaders use best agile practices to set the rhythm of internal processes of working with data. Moreover, they do everything they can to maintain the rhythm of work and, when other team members miss a beat, to improvise accordingly. Finally, they remember their place is at the back of the band, so that others have the freedom to explore the data in whichever way they think creates most value. 

\end{document}
