\providecommand{\main}{..} 
\documentclass[\main/boa.tex]{subfiles}

\begin{document}
	
	\begin{minipage}[t]{0.915\textwidth}
		\center     
		\includegraphics[width=100px]{img/logos.bw/lach.png} 
	\end{minipage}
	\begin{center}
	\Large \textbf {Laboratorium Cyfrowe Humanistyki Uniwersytetu Warszawskiego (LaCh UW)}
	\end{center}
	
	\vskip 0.3cm
	\normalsize 
	To struktura w ramach Uniwersytetu Warszawskiego, wspierająca humanistów i humanistki realizujących cyfrowe projekty naukowe. Organizujemy także liczne warsztaty z~podstawowych dla humanistyki cyfrowych metod i narzędzi. Jak dotąd dotyczyły one przetwarzania danych i tekstów, web scrappingu, tworzenia wydań cyfrowych czy organizacji i budowania nowoczesnych bibliotek oraz repozytoriów. W tym semestrze prowadzimy trzy cykle warsztatowe z podstaw R oraz grupę samokształceniową “Python w badaniach humanistycznych”.
	
	Interesują nas nie tylko metody, ale szersze spojrzenie na kulturę cyfrową. W ramach cyklu “Poza interfejsem” organizujemy wykłady otwarte prowadzone przez osoby, które analizują technologie i oprogramowanie z humanistycznej perspektywy. Podczas tych spotkań sprawdzamy, jak algorytmy wpływają na nasze myślenie o świecie i co kryje się za nieprzejrzystymi interfejsami. Naszymi goścmi byli już między innymi Aleksandra Przegalińska (ALK, MIT) - badaczka zajmująca się teoriami sztucznej inteligencji oraz Paweł Janicki (Centrum Sztuki WRO) - kurator i artysta, twórca interaktywnych instalacji i systemów.
	
	\vskip 1.5cm
\end{document}
