\providecommand{\main}{..} 
\documentclass[\main/boa.tex]{subfiles}

\begin{document}

\section{Kombajn do uczenia maszynowego - MLR w praktyce}

\begin{minipage}{0.915\textwidth}
\centering
{\bf \index[a]{Zawistowski Paweł} Paweł Zawistowski}
\end{minipage}

\vskip 0.3cm

\begin{affiliations}
\begin{minipage}{0.915\textwidth}
\centering
\large Politechnika Warszawska, Wydział EiTI, AdForm  \\[2pt]
\end{minipage}
\end{affiliations}

\vskip 0.8cm

\opiswarsztatu Celem warsztatu jest przedstawienie szerokich możliwości jakie daje pakiet MLR przy tworzeniu różnego rodzaju modeli - przejdziemy przy tym kompletną ścieżkę, od wstępnego przygotowania danych, przez wybór odpowiedniej metody, strojenie hiperparametrów, aż po diagnostykę i wizualizacje wyników.

\planwarsztatu
\begin{enumerate}
\item Ogólne wprowadzenie do pakietu, przygotowanie środowiska MLR.
\item Przygotowanie danych do rozwiązywania naszego zadania (klasyfikacja, regresja, ...).
\item Wybór modelu, strojenie parametrów.
\item Diagnostyka i wizualizacja wyników.
\item Rozszerzanie MLR o własny algorytm.
\item Inne ciekawe elementy pakietu, podsumowanie.
\end{enumerate}	 

\pakiety W ramach warsztatu korzystać będziemy z mlr oraz tidyverse. Udostępniony zostanie również obraz dockera ze wszystkim co potrzebne + RStudio.

\umiejetnosci\begin{enumerate}
	\item Ogólna znajomość zagadnień związanych z tworzeniem modeli statystycznych, umiejętność korzystania z R'a w stylu "tidyverse".
	\item Podstawowa umiejętność korzystania z dockera.
\end{enumerate}

\wymagania Instalacja pakietów R lub ściągnięcie dockera i uruchomienie udostępnionego obrazu.

\end{document}
