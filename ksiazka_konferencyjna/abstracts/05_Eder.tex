\providecommand{\main}{..} 
\documentclass[\main/boa.tex]{subfiles}

\begin{document}

\section{Kim naprawdę był Gall Anonim? Zagadnienia statystycznej analizy tekstu}


\begin{minipage}{0.915\textwidth}
	\centering
  {\bf \LARGE \index[a]{Eder Maciej} Maciej Eder}
\end{minipage}



\begin{affiliations}
\begin{minipage}{0.915\textwidth}
\centering
\large Instytut Języka Polskiego PAN  \\[1pt]
Kontakt: \href{mailto:maciejeder@gmail.com}{\nolinkurl{maciejeder@gmail.com}}\\
\end{minipage}
\end{affiliations}


Wystąpienie będzie poświęcone analizie tekstu za pomocą kilku pakietów języka R, w tym atrybucji autorskiej opartej o statystyczne miary podobieństwa tesktów, a także szeroko rozumianej analizy stylu. Jako jeden z przykładów zostanie omówiony przykład autorstwa "Kroniki polskiej", przypisywanej tzw. Gallowi Anonimowi. W dalszej części wystąpienia zostanie przedstawiona metoda modelowania tematycznego (topic modeling) i jej zastosowania w analizie tekstu. 

\bio
Dr hab. Maciej Eder, profesor Uniwersytetu Pedagogicznego w Krakowie, od kilkunastu lat pracownik Instytutu Języka Polskiego PAN, od roku jego dyrektor. Doktorat (z literatury XVII wieku) obronił w 2005 na Uniwersytecie Wrocławskim, habilitację (z językoznawstwa kwantytatywnego) w 2014 na Uniwersytecie Pedagogicznym.

Zajmuje się analizą danych językowych, w tym modelowaniem zmian w języku staro- i średniopolskim, przede wszystkim zaś stylometrią, czyli kwantytatywną analizą cech językowych, dzięki którym jesteśmy w stanie np. rozpoznać autorstwo anonimowego tekstu. Swoje prace poświęca głównie testowaniu metod wielowymiarowych na materiale różnych języków: polskim, angielskim, łacińskim, starogreckim etc. Jest m.in. autorem rozprawy na temat “Kroniki polskiej” (“Chronica Polonorum”) autorstwa tzw. Galla Anonima, w której weryfikuje hipotezę o weneckim pochodzeniu Galla.

W swojej pracy posługuje się programem R. Jest pomysłodawcą i głównym autorem pakietu “stylo”, zawierającego kilkadziesiąt funkcji do analizy tekstu za pomocą różnych metod stylometrycznych. Wielokrotnie prowadził warsztaty analizy tekstu przy użyciu R w różnych ośrodkach akademickich, m.in. w Lipsku, Victorii, Getyndze, Amsterdamie, Frankfurcie, Padwie, Budapeszcie, Edynburgu. Działa aktywnie w środowisku humanistyki cyfrowej (Digital Humanities), w którym szerzy, z lepszym lub gorszym skutkiem, idee analizy statystycznej w zastosowaniu do danych humanistycznych.

\end{document}
