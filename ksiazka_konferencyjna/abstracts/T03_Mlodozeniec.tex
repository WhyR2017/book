\providecommand{\main}{..} 
\documentclass[\main/boa.tex]{subfiles}

\begin{document}

\subsection{Zastosowanie pakietu Shiny do tworzenia interaktywnych wizualizacji wyników badań eye-trackingowych}

\begin{minipage}{0.915\textwidth}
	\centering
  {\bf \index[a]{Młodożeniec Marek} Marek Młodożeniec}
\end{minipage}


\begin{affiliations}
\begin{minipage}{0.915\textwidth}
\centering
OPI PIB  \\[-2pt]
\end{minipage}
\end{affiliations}

\vskip 0.3cm
Największym ograniczeniem popularnych metod wizualizacji wyników badań \break eye-trackingowych, jakimi są wykresy typu 'scanpath' i 'heatmap', jest ich statyczność. Zwłaszcza te ostatnie nie odzwierciedlają w ogóle funkcji czasu, a jedynie przestrzenne rozłożenie fiksacji, natomiast na wykresach typu 'scanpath' przebieg sakad i fiksacji jest często trudny do prześledzenia. Z kolei tworzenie animacji ukazujących przebieg uwagi wzrokowej wymaga zastosowania specjalistycznych programów, w których \break wygenerowanie animacji wprost z danych surowych bywa problematyczne. Pakiet R Shiny umożliwia tworzenie w prosty sposób interaktywnych aplikacji, które nie tylko pozwalają na śledzenie przebiegu procesu uwagi wzrokowej w formie animacji oraz przewijanie w czasie historii przeszukiwania wzrokowego, ale również dają możliwość regulowania \break parametrów graficznych wizualizacji, tak aby zapewnić jej maksymalną czytelność. Na przykładzie aplikacji R Shiny pokażę, że w zastosowaniu do wizualizacji danych eye-trackingowych pakiet ten tworzy nową jakość, pozwalając na zaprezentowanie odbiorcom informacji trudnych do ukazania na wykresach innego typu. 

\end{document}
