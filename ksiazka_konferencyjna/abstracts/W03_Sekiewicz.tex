\providecommand{\main}{..} 
\documentclass[\main/boa.tex]{subfiles}

\begin{document}

\section{Pakiet rvest, czyli web scrapingu wybrane przypadki}
\begin{minipage}[t]{0.915\textwidth}
	\center     
    \includegraphics[width=60px]{img/workshops/czarno_biale/bartosz-crop.png} 
\end{minipage}

\begin{minipage}{0.915\textwidth}
\centering
{\bf \index[a]{Sękiewicz Bartosz} Bartosz Sękiewicz}
\end{minipage}

\vskip 0.3cm

\begin{affiliations}
\begin{minipage}{0.915\textwidth}
\centering
\large HTA Consulting  \\[2pt]
\end{minipage}
\end{affiliations}

\vskip 0.8cm

\opiswarsztatu Celem warsztatu jest pokazanie z jakimi problemami możemy spotkać się podczas scrapowania stron www przy użyciu pakietu rvest. Warsztat pozwoli uczestnikom na uświadomienie sobie tego jak różnorodne mogą być strony internetowe (w kontekście ich konstrukcji). Dzięki poznaniu niuansów związanych z web scrapingiem możliwe będzie zaoszczędzenie w przyszłości sporej ilości czasu i nerwów. Z uwagi na ograniczoną ilość czasu pominiemy temat scrapowania stron obsługiwanych przez skrypty JS (wymaga to zastosowania dodatkowego oprogramowania jak PhantomJS, lub innego typu webscrapera jak RSelenium).

\planwarsztatu Podczas spotkania postaramy się rozwiązać problemy z pobieraniem danych ze stron zaproponowanych przez uczestników. Skupimy się na trzech aspektach:
\begin{enumerate}
\item piękno języka css, czyli wyciąganie informacji z kodu strony (m.in. tagi, klasy, id, rodzice i dzieci, sąsiedzi);
\item komunikacja ze stronami oraz nawigacja po nich (m.in. formularze, POST i GET);
\item API, czyli jak zaoszczędzić sobie czas (niestety nie zawsze jest to prawdziwe).
\end{enumerate}	 

\end{document}
