\providecommand{\main}{..} 
\documentclass[\main/boa.tex]{subfiles}

\begin{document}

\section{Web scraping w R i nie tylko}

\begin{minipage}{0.915\textwidth}
\centering
{\bf \index[a]{Mazurek Magdalena} Magdalena Mazurek}
\end{minipage}

\vskip 0.3cm

\begin{affiliations}
\begin{minipage}{0.915\textwidth}
\centering
\large Koło Naukowe Data Science  \\[2pt]
\end{minipage}
\end{affiliations}

\vskip 0.8cm

\opiswarsztatu Celem warsztatu jest zaprezentowanie możliwości pakietu RSelenium. Przedstawienie krótko jego wad oraz zalet. Uczestnicy z warsztatów dowiedzą się jak scrapować informacje ze stron internetowych wykorzystujących javascript oraz czemu warto przy tym używać zewnętrznej aplikacji PhantomJS.

\planwarsztatu
Warsztaty rozpoczniemy od zaznajomienia uczestników z zasadą działania RSelenium oraz czym różni się od pakietu rvest. Zaczniemy od korzystania z RSelenium z użyciem klasycznej przeglądrki. W pierwszej kolejności zajmiemy się krótko scrapowaniem stron statycznych, niekorzystajacych z javascriptu jako prezentacja, że tradycyjne scrapowanie jest również możliwe, powiemy jednak czemu jest to nieefektywne. Następnie przejdziemy do części głównej, tj. scrapwowania stron korzystajacych z javascriptu, powiemy w tym miejscu czemu RSelenium jest możliwe do wykonywania tego. Na próbnej stronie pokażemy w jaki sposób korzystamy z pakietu. Na koniec powiemy o możliwości użycia aplikacji PhaontomJS.	 

\pakiety RSelenium

\umiejetnosci Podstawowa znajomość R i HTML.

\wymagania Instalacja aplikacji PhantomJS, najnowszej wersji Java

\end{document}
