\providecommand{\main}{..} 
\documentclass[\main/boa.tex]{subfiles}

\begin{document}

\subsection{Analiza koszykowa, a efektywność promocji w Retailu}

\begin{minipage}{0.915\textwidth}
	\centering
  {\bf \index[a]{Michał Cisek} Michał Cisek, \index[a]{Rafał Kobiela} Rafał Kobiela}
\end{minipage}


\begin{affiliations}
\begin{minipage}{0.915\textwidth}
\centering
PwC \\[-2pt]
\end{minipage}
\end{affiliations}

\vskip 0.3cm

 Podczas wystąpienia przedstawiony zostanie typowy problem analizy koszykowej \break i wpływu promocji na wolumen i marżę sprzedaży w długim okresie czasu. Poruszone zostaną zagadnienia up-sellu, cross-sellu, cherry-pickingu i kanibalizacji sprzedaży. Pokażemy także jak można użyć R'a by podejść do problemu analizy paragonów, który wkracza w obszar Big Data, i jak w prosty sposób można zrównoleglić obliczenia na wiele maszyn połączonych w klaster obliczeniowy na przykładzie infrastruktury konwergentnej. 

\end{document}
