\providecommand{\main}{..} 
\documentclass[\main/boa.tex]{subfiles}

\begin{document}

\section{Klasyfikacja wieloetykietowa z pakietem R}

\begin{minipage}{0.915\textwidth}
\centering
{\bf \index[a]{Teisseyre Paweł} Paweł Teisseyre}
\end{minipage}

\vskip 0.3cm

\begin{affiliations}
\begin{minipage}{0.915\textwidth}
\centering
\large Instytut Podstaw Informatyki PAN  \\[2pt]
\end{minipage}
\end{affiliations}

\vskip 0.8cm

\opiswarsztatu Celem warsztatów jest przedstawienie problemu klasyfikacji wieloetykietowej oraz pokazanie jak wykorzystać R do modelowania danych z wieloma etykietami. W klasycznym problemie klasyfikacji modelujemy zależność między zmienną odpowiedzi (najczęściej binarną) a zmiennymi objaśniającymi. W klasyfikacji wieloetykietowej rozważamy wiele binarnych zmiennych odpowiedzi jednocześnie. W ostatnich latach klasyfikacja wieloetykietowa wzbudziła bardzo duże zainteresowanie. Metody klasyfikacji wieloetykietowej są stosowane w wielu dziedzinach, takich jak automatyczna kategoryzacja tekstów, rozpoznawanie obrazów, modelowanie wielozachorowalności (współwystępowanie wielu chorób jednocześnie), przewidywanie skutków ubocznych leków i wiele innych. Podczas warsztatów opowiem o popularnych metodach stosowanych w klasyfikacji wieloetykietowej (takich jak łańcuchy klasyfikatorów). Podczas części praktycznej zajmiemy się analizą rzeczywistych zbiorów danych.

\planwarsztatu
\begin{enumerate}
\item Teoria (omówienie problemu, przegląd metod).
\item Analiza danych rzeczywistych
\end{enumerate}	 

\pakiety mldr

\umiejetnosci Podstawowa znajomość metod klasyfikacji i regresji.

\wymagania Brak

\end{document}
