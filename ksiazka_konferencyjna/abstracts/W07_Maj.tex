\providecommand{\main}{..} 
\documentclass[\main/boa.tex]{subfiles}

\begin{document}

\section{Machine Learning w R przy użyciu H2O}

\begin{minipage}{0.915\textwidth}
\centering
{\bf \index[a]{Maj Michał} Michał Maj}
\end{minipage}

\vskip 0.3cm

\begin{affiliations}
\begin{minipage}{0.915\textwidth}
\centering
\large Appsilon Data Science, trigeR  \\[2pt]
\end{minipage}
\end{affiliations}

\vskip 0.8cm

\opiswarsztatu Celem warsztatu jest zapoznanie uczestników z platformą H2O oraz dostępnymi algorytmami uczenia maszynowego jak np. Generalized linear model (GLM), Gradient Boosted Machines (GBM), Deep Neural Networks (DNN), K-means, Ensemble Methods.

\planwarsztatu
\begin{enumerate}
\item Wprowadzenie - czym jest H2O i jak działa?
\item Przygotowanie i transformacje danych w H2O
\item Przegląd algorytmów + przykłady
\item Tuningowanie parametrów modelu
\item Ensemble Methods
\item Podsumowanie i dalsze wskazówki
\end{enumerate}	 

\pakiety h2o, dplyr, ggplot2, h2oEnsemble

\umiejetnosci Podstawowa znajomość języka R. Mile widziana (choć niekonieczna) podstawowa wiedza z zakresu algorytmów uczenia maszynowego.

\wymagania R, RStudio, wymagane pakiety.

\end{document}
