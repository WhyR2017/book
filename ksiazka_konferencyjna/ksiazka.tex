\documentclass[11pt,twoside,b5paper]{book}

\usepackage[b5paper,tmargin=1cm,bmargin=1.5cm, innermargin=2cm]{geometry}

\usepackage{fancyhdr}
\pagestyle{plain}
\fancyhf{}


\fancypagestyle{mainmatter}{
\rhead{Why R? 2017}
\lhead{Książka konferencyjna}
\rfoot{\thepage}
}

\usepackage{emptypage}
\usepackage{enumitem}

\usepackage{subfiles}
\usepackage{hyperref}
\usepackage{titlesec}

\usepackage{longtable} 
\usepackage{pdflscape} 
\usepackage{afterpage}
\usepackage{parskip}

\usepackage{graphicx}
\usepackage{caption}
\usepackage{subcaption}

\usepackage{imakeidx}
\makeindex[name=a,title=Indeks nazwisk,intoc=true]

\usepackage[utf8]{inputenc}
\usepackage[T1]{fontenc}
\usepackage{eurosym}
\usepackage{amsfonts, amsmath, hanging, hyperref, parskip, times}
\usepackage[numbers]{natbib}
\usepackage[pdftex]{graphicx}
\hypersetup{
	colorlinks,
	linkcolor=black,
	urlcolor=black,
	citecolor=black
}

\usepackage{pdfpages}

%\let\section=\subsubsection
\newcommand{\pkg}[1]{{\normalfont\fontseries{b}\selectfont #1}}
\let\proglang=\textit
\let\code=\texttt
\newcommand{\atitle}[1]{\begin{center}{\bf \LARGE #1}\end{center}}
\newcommand{\affiliations}{\footnotesize\centering}
\newcommand{\keywords}{\paragraph{Keywords:}}
\newcommand{\opiswarsztatu}{\subsubsection{Opis warsztatu}\noindent \small}
\newcommand{\planwarsztatu}{\subsubsection{Plan warsztatu}\small}
\newcommand{\pakiety}{\subsubsection{Wymagane pakiety}\noindent \small}
\newcommand{\umiejetnosci}{\subsubsection{Wymagane od uczestników umiejętności i wiedza}\noindent \small}
\newcommand{\wymagania}{\subsubsection{Wymagania wstępne do wykonania przed warsztatem}\noindent \small}
\newcommand{\sylwetkaprowadzacego}{\subsubsection{Sylwetka prowadzącego}\noindent \small}
\newcommand{\packages}{\subsubsection{R packages:}}
\newcommand{\bio}{\subsubsection{Bio} \small}


\renewcommand*\contentsname{Spis treści}

\newcommand{\mychapter}[1]{\chapter{#1} \title{#1} \newpage}

\usepackage{titlesec}
\titleformat{\chapter}[display]   
{\normalfont\Huge\bfseries}{\chaptertitlename\ \thechapter}{80pt}{\LARGE}   
\titlespacing*{\chapter}{0pt}{0pt}{0pt}

\providecommand{\tightlist}{%
	\setlength{\itemsep}{0pt}\setlength{\parskip}{0pt}}

% \setlength{\topmargin}{-25mm}
\setlength{\oddsidemargin}{-2mm}
% \setlength{\textwidth}{165mm}
% \setlength{\textheight}{250mm}

\titleformat{\chapter}[display]
{\normalfont\bfseries}{}{0pt}{\huge}

\titleformat{\section}[block]
{\LARGE\bfseries}{}{0pt}{\titlerule\\}[\vspace{2pt}\titlerule]

\titlespacing\subsubsection{0pt}{12pt plus 4pt minus 2pt}{0pt plus 2pt minus 2pt}
%\titleformat{\section}[block]
%{\Large\bfseries\filcenter}{}{1em}{}

\setcounter{secnumdepth}{0}
\setcounter{tocdepth}{2}

\title{Konferencja Why R?}
\author{Warszawa, 27-29 września 2017}
\date{}

\begin{document}

\includepdf{figs/okladka.pdf}
% \cleardoublepage

\clearpage

\frontmatter
\maketitle

\begin{small}
\cleardoublepage
\pagenumbering{gobble}
\tableofcontents
\cleardoublepage
\pagenumbering{arabic} 
\end{small}

\mainmatter

\chapter*{Komitet organizacyjny}
\par
\begin{figure}[h!]
\includegraphics[width=\linewidth]{img/people/all.png}
\end{figure}


\chapter*{Wstęp}
 Lorem ipsum dolor sit amet, consectetur adipiscing elit. Sed convallis ut magna at egestas. Vestibulum feugiat lobortis leo ut accumsan. Phasellus sed mattis nibh. Interdum et malesuada fames ac ante ipsum primis in faucibus. Nulla ultrices leo et turpis facilisis ullamcorper. In in augue scelerisque, aliquam magna id, laoreet lacus. Nam pulvinar ipsum vel diam sodales, nec gravida dui ultrices. Donec finibus sit amet ex ac dictum. Suspendisse potenti. Nam varius turpis et tincidunt iaculis. Aenean vel mauris eget diam dictum sodales ac quis arcu. Donec in augue id nisi mattis ultrices.
 
 Cras porttitor, tellus vel auctor dignissim, diam elit bibendum nunc, at tempor ipsum leo vitae elit. Quisque tincidunt nisi sapien, eget pretium ipsum scelerisque in. Mauris porttitor, lacus vitae molestie placerat, justo elit consectetur ante, non bibendum mi tortor eu tellus. Proin luctus elit mi, vel suscipit sem feugiat egestas. Nunc at nisl id ex blandit finibus viverra a libero. Cras faucibus libero in nulla eleifend, sit amet iaculis sapien congue. Aliquam id lacus vel ex posuere maximus sed a massa. Nullam sed lacus at erat tristique dictum. Curabitur consectetur ligula tortor, eget varius massa vestibulum a. Suspendisse potenti.
 
 Suspendisse hendrerit aliquam laoreet. Nunc tempor, ligula sit amet consequat convallis, odio mi tempor justo, varius sodales metus sem vel arcu. Cras nec felis urna. Etiam egestas sagittis tellus, a egestas felis lacinia ut. Proin odio velit, ullamcorper lacinia condimentum a, bibendum vitae enim. Proin feugiat gravida turpis, cursus suscipit risus vehicula a. Donec posuere eros ipsum, in venenatis lectus imperdiet eget. Nam vitae ullamcorper nisl. Cras luctus id diam pretium fringilla. Etiam eleifend urna ac erat laoreet, eget scelerisque nunc hendrerit. Aenean ac vulputate risus. Mauris id rutrum mauris. Ut viverra a nisi ac cursus. Vestibulum auctor, tortor et tempor malesuada, libero tellus egestas nulla, id facilisis nulla massa in ipsum. Praesent iaculis porttitor velit, eget eleifend turpis sollicitudin quis. Mauris volutpat metus sodales porttitor tempor. 


\chapter*{Plan konferencji}
\noindent\makebox[\textwidth]{\includegraphics[width=1.25\textwidth,trim=0cm 10cm 0cm 0cm,clip]{figs/plan.pdf}}

\chapter*{Plan warsztatów}
\noindent\makebox[\textwidth]{\includegraphics[width=1.25\textwidth,trim=0cm 10cm 0cm 0cm,clip]{figs/plan_warsztaty.pdf}}

\chapter{Wykłady plenarne}{}
\subfile{abstracts/04_Burzykowski.tex}
\newpage
\subfile{abstracts/09_Szczurek.tex}
\newpage
\subfile{abstracts/06_Jakuczun.tex}
\newpage
\subfile{abstracts/08_Suchwalko.tex}
\newpage
\subfile{abstracts/02_Bogdan.tex}
\newpage
\subfile{abstracts/07_Ramsza.tex}
\newpage
\subfile{abstracts/05_Eder.tex}
\newpage
\subfile{abstracts/03_Brzezinska.tex}
\newpage
\subfile{abstracts/01_Biecek.tex}
\newpage
\subfile{abstracts/10_Wroblewska.tex}







\chapter{Sesje}{\LARGE}
\section{Tidyverse}{}
\subfile{abstracts/T01_Mierzwa.tex}
\subfile{abstracts/T02_Rogala.tex}
\subfile{abstracts/T03_Mlodozeniec.tex}
\subfile{abstracts/T04_Potocka.tex}
\subfile{abstracts/T05_Sobczyk.tex}
\newpage
\section{PwC Data Analytics}
PwC to czwarta najsilniejsza marka świata według badania Brand Finance, a jednocześnie zwycięzca w 3 rankingach Warsaw Busiuness Journal: doradztwa biznesowego, podatkowego oraz usług audytowo- księgowych.

Firma PwC kieruje się w swojej działalności trzema głównymi wartościami: jakością i doskonałością, pracą zespołową oraz przywództwem. Świadczy usługi z zakresu audytu, doradztwa biznesowego, podatkowego i prawnego, jak również w obszarze digital transformation. Jest obecna w 157 krajach zatrudniając ponad 233 tysięcy pracowników na całym świecie.

W Polsce PwC zatrudnia zespół blisko 3 500 pracowników w ośmiu miastach: w Gdańsku, Katowicach, Krakowie, Łodzi, Poznaniu, Rzeszowie, Wrocławiu i Warszawie.
\subfile{abstracts/PC01_Pitera.tex}
\subfile{abstracts/PC02_Chmura.tex}\newpage
\subfile{abstracts/PC03_Kosinski.tex}
\newpage
\section{Statystyka}
\subfile{abstracts/MS01_Wojcik.tex}
\subfile{abstracts/MS02_Dyderski.tex}
\subfile{abstracts/MS03_Nowacki.tex}
\subfile{abstracts/MS04_Slomczynski.tex}
\subfile{abstracts/MS05_Grala.tex}
\newpage
\section{Modelowanie Ryzyka (UBS)}
UBS jest jedną z największych instytucji finansowych na świecie, z ponad 150-letnią historią. Działalność firmy koncentruje się na trzech głównych obszarach: zarządzaniu majątkiem, zarządzaniu aktywami oraz bankowości inwestycyjnej. Zatrudniamy ponad 60000 osób w ponad 50-ciu krajach. Główna siedziba UBS znajduje się w Szwajcarii, natomiast w biurach w Krakowie i we Wrocławiu pracownicy współpracują w ramach zespołów zlokalizowanych w różnych regionach Europy i świata.

UBS tworzy środowisko ludzi zafascynowanych modelowaniem, którzy chętnie dzielą się swoją wiedzą, a język R jest szeroko używany wewnątrz UBS w obszarze modelowania ryzyka.
\subfile{abstracts/MR01_Ochotny.tex}
\subfile{abstracts/MR02_Wrobel.tex}
\subfile{abstracts/MR03_Kowalczyk.tex}
\newpage
\section{Biostatystyka}
\subfile{abstracts/BIO01_Stankiewicz.tex}
\subfile{abstracts/BIO02_Gosiewska.tex}
\subfile{abstracts/BIO03_Dabrowska.tex}
\subfile{abstracts/BIO04_Kosinski.tex}
\subfile{abstracts/BIO05_Oles.tex}
\newpage
\section{Biznes}
\subfile{abstracts/BIZ01_Olszewski.tex}
\subfile{abstracts/BIZ02_Skrzydlo.tex}\newpage
\subfile{abstracts/BIZ03_Zoltak.tex}
\subfile{abstracts/BIZ04_Jedrzejewski.tex}
\subfile{abstracts/BIZ05_Martsenyuk.tex}
\newpage
\section{Społeczności}
\textbf{Dla kogo ?} \\
Dla wszystkich zaangażowanych w tworzenie społeczności R w Polsce. Zarówno dla osób, które takie grupy prowadzą, jak również dla tych którzy chcą utworzyć społeczność entuzjastów R w swojej okolicy :)

\textbf{Dlaczego ?}\\
Grupy pasjonatów odgrywają ogromną rolę w popularyzacji R. Celem każdego ze spotkań jest wymiana wiedzy i doświadczeń pomiędzy obecnymi i nowymi użytkownikami, kształcenie oraz networking. W Polsce aktywnie działa 6 grup. W trakcie sesji chcielibyśmy pomóc w znalezieniu inspiracji do dalszego działania oraz ułatwić wymianę doświadczeń związanych z organizacją społeczności skupiających entuzjastów R w Polsce.

\textbf{Czego można się spodziewać ?}\\
Żywiołowej dyskusji pomiędzy przedstawicielami istniejących społeczności R w Polsce, pod kontrolą charyzmatycznego prowadzącego :)
\newpage
\section{R w działaniu}
\subfile{abstracts/R01_Otmianowski.tex}
\subfile{abstracts/R02_Melcer.tex}
\subfile{abstracts/R04_Bigos.tex}\newpage
\subfile{abstracts/R05_Kochanski.tex}
\newpage
\chapter{Lightning Talks}{\LARGE \textit{Sala 107 - Chairman: Marcin Kosiński}}
\subfile{abstracts/L01_Chmura.tex}
\subfile{abstracts/L02_Czernecki.tex}
\subfile{abstracts/L03_Bielski.tex}
\subfile{abstracts/L05_Lodzikowski.tex}
\subfile{abstracts/L06_Staniak.tex}
\newpage
\chapter{Sesja plakatowa}{\LARGE \textit{Aula Gmachu Fizyki PW}}
\subfile{abstracts/P01_Pawlik.tex}
\subfile{abstracts/P02_Czortek.tex}
\subfile{abstracts/P03_Pawlowska.tex}
\newpage
\chapter{Warsztaty}{}
\subfile{abstracts/W01_Szklarczyk.tex}
\newpage
\subfile{abstracts/W02_Zoltak.tex}
\newpage
\subfile{abstracts/W03_Sekiewicz.tex}
\newpage
\subfile{abstracts/W04_Mazurek.tex}
\newpage
\subfile{abstracts/W05_Grala.tex}
\newpage
\subfile{abstracts/W06_Detko.tex}
\newpage
\subfile{abstracts/W07_Maj.tex}
\newpage
\subfile{abstracts/W08_Zawistowski.tex}
\newpage
\subfile{abstracts/W09_Alekseichenko.tex}
\newpage
\subfile{abstracts/W10_Teisseyre.tex}
\newpage
\subfile{abstracts/W11_Ocalewicz.tex}
\newpage
\subfile{abstracts/W12_Cwiakowski.tex}
\newpage
\subfile{abstracts/W13_Dyderski.tex}
\newpage
\subfile{abstracts/W14_Migdal.tex}
\newpage
\subfile{abstracts/W15_Wojtasiewicz.tex}
\newpage
\subfile{abstracts/W16_Ryciak.tex}
\newpage
\subfile{abstracts/W17_Zawadzki.tex}
\newpage
\subfile{abstracts/W18_Tartanus.tex}

\backmatter
\small \printindex[a]

\clearpage\phantom{}
 \thispagestyle{empty}
\clearpage\phantom{}
 \thispagestyle{empty}

\includepdf{figs/okladka.pdf}

\end{document}
