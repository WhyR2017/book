\providecommand{\main}{..} 
\documentclass[\main/boa.tex]{subfiles}

\begin{document}

\section{SparkR - wydajne obliczenia w chmurze}

\begin{minipage}{0.915\textwidth}
	\centering
  {\bf \index[a]{Bielski Włodzimierz} Włodzimierz Bielski}
\end{minipage}

\vskip 0.3cm

\begin{affiliations}
\begin{minipage}{0.915\textwidth}
\centering
ITMAGINATION  \\[-2pt]
\end{minipage}
\end{affiliations}

\vskip 0.8cm

 Połączenie R z łatwo dostępną mocą obliczeniową w chmurze pozwala na realizowanie nowych, niedostępnych do tej pory scenariuszy. Pokażę jak prosto możemy sięgnąć po tę moc, korzystając z pakietu SparkR. 

\end{document}
