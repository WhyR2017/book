\providecommand{\main}{..} 
\documentclass[\main/boa.tex]{subfiles}

\begin{document}

\section{Uprzyjemnij sobie generowanie wielu wykresów za pomocą purrr}

\begin{minipage}{0.915\textwidth}
	\centering
  {\bf \index[a]{Otmianowski Mateusz}  Mateusz Otmianowski}
\end{minipage}



\begin{affiliations}
\begin{minipage}{0.915\textwidth}
\centering
Pearson \\[-2pt]
\end{minipage}
\end{affiliations}

\vskip 0.3cm

Praca analityka wymaga tworzenia wielu wykresów, szczególnie w fazie eksploracji danych. Jest to często żmudne zajęcie, które można jednak usprawnić poprzez wykorzystanie pakietu purrr. Zademonstruję jak używać purrr w kombinacji z plotly do hurtowego generowania wykresów oraz przetrzymywania ich w data frame’ach, co ogranicza nakład pracy oraz sprawia, że kod jest zwięzły i zrozumiały dla pozostałych analityków. 

\end{document}
