\providecommand{\main}{..} 
\documentclass[\main/boa.tex]{subfiles}

\begin{document}

\subsection{R a dane w chmurze}

\begin{minipage}{0.915\textwidth}
	\centering
  {\bf \index[a]{Jędrzejewski Krzysztof} Krzysztof Jędrzejewski}
    {\bf \index[a]{Pankowska Emilia} Emilia Pankowska}
\end{minipage}



\begin{affiliations}
\begin{minipage}{0.915\textwidth}
\centering
Pearson \\[-2pt]
\end{minipage}
\end{affiliations}

\vskip 0.3cm

W międzynarodowej korporacji sprawne działanie zespołów analitycznych rozsianych po całym świecie wymaga efektywnego udostępniania, współdzielenia oraz przetwarzania danych. Jedną z możliwości osiągnięcia tego celu w prosty sposób jest skorzystanie z usług chmurowych świadczonych przez specjalizujące się w tym firmy. Jednym z najpopularniejszych dostawców rozwiązań w tym obszarze jest firma Amazon. Na przykładzie jednego z naszych projektów analitycznych opiszemy w jaki sposób z poziomu języka R można przetwarzać dane z wykorzystaniem usług amazonowych. Przedstawimy kilka podejść jakie przetestowaliśmy, ze szczególnym uwzględnieniem ich zalet i wad, oraz problemów jakie napotkaliśmy. 

\end{document}
