\providecommand{\main}{..} 
\documentclass[\main/boa.tex]{subfiles}

\begin{document}

\section{Social Network Analysis w R}

\begin{minipage}{0.915\textwidth}
\centering
{\bf \index[a]{Wojtasiewicz Michał} Michał Wojtasiewicz}
\end{minipage}

\vskip 0.3cm

\begin{affiliations}
\begin{minipage}{0.915\textwidth}
\centering
\large Instytut Podstaw Informatyki PAN  \\[2pt]
\end{minipage}
\end{affiliations}

\vskip 0.8cm

\opiswarsztatu Celem warsztatu jest zaznajomienie uczestników z tematyką analizy danych w sieciach społecznych. Ćwiczenia praktyczne zostaną przeprowadzone w głównej mierze przy użyciu pakietu igraph. Dziedzina Social Network Analysis (SNA) jest teraz jedną z najprężniej rozwijających się dziedzin uczenia maszynowego. Z racji powszechności występowania sieci społecznych (np. Facebook, Instagram, LinkedIn, problemy optymalizacyjne, sieci systemów rekomendujących, sieć połączeń mailowych) zapotrzebowanie na algorytmy i coraz bardziej zaawansowane rozwiązania stale wzrasta. Naturalną metodą zapisu sieci jest graf czyli zbiór wierzchołków i krawędzi. Dzięki łatwej w konstrukcji strukturze, zapis grafowy pozwala na skuteczne rozwiązywanie szerokiego zakresu problemów data miningowych. Na zajęciach warsztatowych uczestnicy zapoznają się z tematyką SNA, głównym problemami oraz popularnymi rozwiązaniami tych problemów. Nauczą się wyznaczać grupy podobnych elementów sieci (np. grupa znajomych), kluczowe ze względu przesyłania informacji elementy sieci (np. bottlenecki), podgrupy elementów pozornie niepowiązanych (np. grupa klientów kupujących ten sam produkt) oraz użycia sieci do szeregowania zadań (kolorowanie zwarte).

\planwarsztatu
\begin{enumerate}
\item Wprowadzenie do tematyki SNA.
\item Omówienie przykłądowej sieci poprzez strukturę grafu.
\item Analiza skupień w sieci społecznej.
\item Wyznaczenie różnych rodzajów najbardziej wpływowych elementów sieci.
\item Wprowadzenie do systemów rekomendujących.
\item Szeregowania zadań z restrykacją braku przestoju.
\end{enumerate}	 

\pakiety igraph", Matrix, visNetwork

\umiejetnosci Podstawowa znajomość języka R.

\wymagania Pobranie przykładowego grafu oraz wymaganych pakietów.

\end{document}