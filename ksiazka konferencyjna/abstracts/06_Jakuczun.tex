\providecommand{\main}{..} 
\documentclass[\main/boa.tex]{subfiles}

\begin{document}

\section{R jako główna platforma do zaawansowanej analityki w Enterprise}


\begin{minipage}{0.915\textwidth}
	\centering
  {\bf \huge \index[a]{Jakuczun Wit} Wit Jakuczun}
\end{minipage}


\vskip 0.3cm

\begin{affiliations}
\begin{minipage}{0.915\textwidth}
\centering
\large WLOG Solutions \\[5pt]
Kontakt: \href{mailto:wit.jakuczun@wlogsolutions.com}{\nolinkurl{wit.jakuczun@wlogsolutions.com}}\\
\end{minipage}
\end{affiliations}

\vskip 0.8cm

Świat hermetycznych platform analitycznych powoli staje się historią. Dzisiaj analityka zaawansowana jest pchana do przodu przez świat open-source wspierany przez największych graczy. W różnych dyskusjach stawiane jest pytanie o dojrzałość R z punktu widzenia wymagań korporacji. Na podstawie wdrażania R w dużym telekomie opowiem dlaczego uważam, że R może być numerem jeden jeśli chodzi o zaawansowaną analitykę w każdej dużej korporacji. Pokażę na co zwrócić uwagę i jakie są plusy i minusy przejścia na R. 

\bio
Wit Jakuczun to założyciel i współwłaściciel firmy doradczej WLOG Solutions będącej strategicznym partnerem we wdrażaniu rozwiązań analitycznych dużej skali w oparciu o środowisko R. W firmie jest odpowiedzialny za tłumaczenie potrzeb biznesowych klientów na język matematyki.

Prowadził projekty dla wielu branż: bankowość, energetyka, gaz, marketing, logistyka, farmacja, retail, telekomunikacja, ubezpieczenia. W ramach tych projektów tworzył i wdrażał rozwiązania wielkiej skali w środowisku R (i nie tylko) wykorzystujące modele predykcyjne, optymalizacyjne i symulacyjne.

\end{document}
