\providecommand{\main}{..} 
\documentclass[\main/boa.tex]{subfiles}

\begin{document}

\subsection{Czy R (kiedyś) zastąpi SPSS?}

\begin{minipage}{0.915\textwidth}
	\centering
  {\bf \index[a]{Zółtak Tomasz}  Tomasz Zółtak}
\end{minipage}



\begin{affiliations}
\begin{minipage}{0.915\textwidth}
\centering
Instytut Badań Edukacyjnych \\[-2pt]
\end{minipage}
\end{affiliations}

\vskip 0.3cm

 R staje się coraz popularniejszym narzędziem również w dziedzinie nauk społecznych, w tym w analizie danych sondażowych. Ogromne znaczenie dla tego obszaru zastosowań miał rozwój możliwości związanych z tworzeniem raportów, jaki nastąpił w ostatnich latach. Niemniej R wciąż nie jest wymarzonym narzędziem do "robienia tabelek" prezentujących wyniki typowych sondaży. W wystąpieniu zamierzam zarysować specyficzne problemy związane z analizą danych sondażowych oraz zastanowić się, jakie rozwiązania niezbędne do wygodnego prowadzenia takich analiz są już dostępne w środowisku R, a jakich elementów moim zdaniem wciąż brakuje. 

\end{document}
