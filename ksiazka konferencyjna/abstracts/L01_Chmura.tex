\providecommand{\main}{..} 
\documentclass[\main/boa.tex]{subfiles}

\begin{document}

\section{Lasy z inwazyjnym dębem czerwonym w świetle analiz wielowymiarowych przy użyciu R }

\begin{minipage}{0.915\textwidth}
	\centering
  {\bf \index[a]{Chmura Damian} Damian Chmura }
\end{minipage}

\vskip 0.3cm

\begin{affiliations}
\begin{minipage}{0.915\textwidth}
\centering
Akademia Techniczno-Humanistyczna w Bielsku-Białej  \\[-2pt]
\end{minipage}
\end{affiliations}

\vskip 0.8cm

 Północnoamerykański dąb czerwony Quercus rubra L. zaliczany jest do tzw. gatunków inwazyjnych w naszej florze, tzn. rozprzestrzenia się spontanicznie i wywiera negatywny wpływ na rodzime gatunki roślin występujące głównie w runie. Celem niniejszych badań była analiza wielowymiarowa (analiza skupień, techniki ordynacyjne i analiza funkcjonalna) składu gatunkowego lasów zastępczych z udziałem dębu czerwonego. W latach 2008-2011 wykonano 180 zdjęć fitosocjologicznych w wybranych losowo kompleksach leśnych z udziałem dębu na Wyżynie Śląskiej. Zebrany materiał poddano analizie skupień. Ze względu na to, że miara odległości ma wpływ na końcowy wynik zastosowano funkcję rankindex (pakiet vegan) a jako matrycę danych siedliskowych użyto średnich arytmetycznych i ważonych liczb Ellenberga. W celu ustalenia kierunków zmienności uzyskane jednostki roślinności przeanalizowano technikami ordynacyjnymi (nietendencyjna analiza zgodności, DCA). Wpływ wybranych czynników siedliskowych na zmienność składu gatunkowego oraz pokrycie dębu w poszczególnych warstwach (warstwa drzew, podszyt i runo) sprawdzono przy użyciu kanonicznej analizy korespondencji CCA, analizy redundancji RDA oraz testów permutacyjnych, pakiety: vegan, ade4). Zastosowano analizę indVal (indicator value), aby stwierdzić czy są istotne statystycznie gatunki wskaźnikowe dla wybranych typów lasów (pakiety: labdsv, indicspecies). Wykonano również analizę taksonomiczną i funkcjonalną analizowanych lasów. Wyliczone wskaźniki: bogactwa gatunkowego i różnorodności gatunkowej (wskaźnik Shannona-Wienera i in.) oraz różnorodności funkcjonalnej (pakiet FD: funkcjonalne bogactwo FRic, funkcjonalna jednorodność FEve i funkcjonalna rozbieżność FDiv) porównano między badanymi lasami. Wyniki analiz wielowymiarowych pozwalają na określenie, na jakich typach siedlisk dąb czerwony częściej dokonuje inwazji lub wcześniej był sadzony, z jakimi gatunkami współwystępuje oraz jaki jest wpływ tego drzewa na rośliny towarzyszące. 

\end{document}
