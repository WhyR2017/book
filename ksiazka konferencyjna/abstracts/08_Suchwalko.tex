\providecommand{\main}{..} 
\documentclass[\main/boa.tex]{subfiles}

\begin{document}

\section{Jak dużo mocy (i po co) można wycisnąć z modelu predykcyjnego?}


\begin{minipage}{0.915\textwidth}
	\centering
  {\bf \huge \index[a]{Suchwałko Artur} Artur Suchwałko}
\end{minipage}


\vskip 0.3cm

\begin{affiliations}
\begin{minipage}{0.915\textwidth}
\centering
\large QuantUp \\[5pt]
Kontakt: \href{mailto:artur@quantup.pl}{\nolinkurl{artur@quantup.pl}}\\
\end{minipage}
\end{affiliations}

\vskip 0.8cm

W modelowaniu predykcyjnym często wybieramy bardzo złożone podejścia i modele. Z drugiej strony, często też stosuje się podejścia do modelowania, które wręcz jest wstyd stosować w dzisiejszych czasach.

Predykcję można poprawić na różne sposoby, na przykład poprzez wykorzystanie bardziej złożonych modeli, staranny dobór hiperparametrów, uwzględnienie kosztów błędnej klasyfikacji czy zmianę kryterium optymalizacji.

Pokażę na przykładzie, co to daje dla biznesu oraz jak to zrobić w R.

\end{document}
