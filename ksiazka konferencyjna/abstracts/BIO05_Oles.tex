\providecommand{\main}{..} 
\documentclass[\main/boa.tex]{subfiles}

\begin{document}

\subsection{Podstawy przetwarzania i analizy obrazów w R }

\begin{minipage}{0.915\textwidth}
	\centering
  {\bf \index[a]{Oleś Andrzej} Andrzej Oleś}
\end{minipage}

\vskip 0.3cm

\begin{affiliations}
\begin{minipage}{0.915\textwidth}
\centering
EMBL Heidelberg \\[-2pt]
\end{minipage}
\end{affiliations}

\vskip 0.8cm

W oparciu o pakiet do przetwarzania i analizy obrazów EBImage zademonstrowane zostaną metody pracy z danymi graficznymi w R: wczytywanie i wyświetlanie, transformacje przestrzenne oraz filtrowanie. Na przykładzie mikroskopowych obrazów komórek pokazane zostanie jak przeprowadzić segmentację obrazu w celu wyodrębnienia charakterystyk ilościowych obiektów, stanowiących punkt wyjścia do dalszych analiz statystycznych. 

\end{document}
