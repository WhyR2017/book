\providecommand{\main}{..} 
\documentclass[\main/boa.tex]{subfiles}

\begin{document}

\section{Nie pisz kodu, pisz prozę - wprowadzenie do pakietu dplyr}

\begin{minipage}{0.915\textwidth}
\centering
{\bf \index[a]{Tartanus Bartłomiej} Bartłomiej Tartanus}
\end{minipage}

\vskip 0.3cm

\begin{affiliations}
\begin{minipage}{0.915\textwidth}
\centering
\large OSA/Sages  \\[2pt]
\end{minipage}
\end{affiliations}

\vskip 0.8cm

\opiswarsztatu Ramki danych w R (data.frame) są niezbędne do pracy z danymi w postaci tabelarycznej. Jednak często przetwarzanie takich ramek przy użyciu czystego R prowadzi do wielokrotnego powtarzania nazw ramki czy kolum w celu np przefiltrowania wierszy lub niewygodnego doklejania kolumn w przypadku rozszerzania ramki. Taki kod często bywa nieczytelny i nie do końca oddaje intencje autora. Do takiego kodu ciężko wrócić za jakiś czas i przypomnieć sobie "co ja tutaj chciałem zrobić". Wtedy mamy faktycznie do czynienia z "kodem". Czy tak musi być już zawsze? Na szczęście nie. Z pomocą przychodzi pakiet dplyr, który pozwoli nam pisać "prozę" - nasz kod będzie o wiele czytelniejszy.	 

\pakiety dplyr

\umiejetnosci Podstawowa znajomość R - operacje na wektorach, tworzenie ich oraz znajomość ramek danych.

\wymagania R, RStudio

\end{document}