\providecommand{\main}{..} 
\documentclass[\main/boa.tex]{subfiles}

\begin{document}

\section{Analiza danych obserwacyjnych zwiedzających wystawy w Centrum Nauki Kopernika}


\begin{minipage}{0.915\textwidth}
	\centering
  {\bf \huge \index[a]{Wróblewska Anna} Anna Wróblewska}
\end{minipage}


\vskip 0.3cm

\begin{affiliations}
\begin{minipage}{0.915\textwidth}
\centering
\large MiNI, Politechnika Warszawska  \\[5pt]
Kontakt: \href{mailto:awroble@gmail.com}{\nolinkurl{awroble@gmail.com}}\\
\end{minipage}
\end{affiliations}

\vskip 0.8cm
W prezentacji podsumowujemy współpracę z Centrum Nauki Kopernik (CNK) (Katarzyna Potęga), Uniwersytetem Nauk Społecznych i Humanistycznych (Łukasz Tanaś) oraz Wydziałem Matematyki i Nauk Informacyjnych Politechniki Warszawskiej (WUT). Pracowaliśmy nad danymi obserwacyjnymi zebranymi podczas testów przeprowadzonych w CNK. Dane te zostały przeanalizowane przez studentów nowej specjalności Przetwarzanie i analiza danych na wydziale MINI PW.
Dane zostały zebrane z trzech badań obserwacyjnych dotyczących zachowania dzieci i rodziców oraz dzieci szkolnych, a także testów dotyczących rozpoznawania i zrozumienia emocji.
Postawiliśmy wiele hipotez i pytań badawczych, zweryfikowaliśmy je w oparciu o dostępne dane, np. podsumujemy różne postawy rodziców, a także zaufanie i rozpoznanie emocji oraz zaangażowanie dzieci w oglądane/doświadczane eksponaty. 

\end{document}
