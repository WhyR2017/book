\providecommand{\main}{..} 
\documentclass[\main/boa.tex]{subfiles}

\begin{document}

\section{Dezagregacja danych przedziałowych}


\begin{minipage}{0.915\textwidth}
	\centering
  {\bf \huge \index[a]{Ramsza Michał} Michał Ramsza}
\end{minipage}


\vskip 0.3cm

\begin{affiliations}
\begin{minipage}{0.915\textwidth}
\centering
\large Szkoła Główna Handlowa w Warszawie \\[5pt]
Kontakt: \href{mailto:michal.ramsza@gmail.com}{\nolinkurl{michal.ramsza@gmail.com}}\\
\end{minipage}
\end{affiliations}

\vskip 0.8cm

Zostanie przedstawiona metoda dezagregacji danych przedziałowych oraz jej implementacja w języku R. 

\bio
Michał Ramsza uzyskał stopień magistra w zastosowaniach matematyki (1996) na Uniwersytecie Warszawskim, stopień doktora nauk ekonomicznych (2000) oraz stopień doktora habilitowanego (2010) w Szkole Głównej Handlowej w Warszawie.

Pracował w różnych bankach jako analityk, dyrektor Departamentu Analiz Rynkowych w KNF, ale również jako research fellow w University College London. Współpracuje z Instytutem Badań Strukturalnych.

Obecnie pracuje jako profesor ekonomii matematycznej w Szkole Głównej Handlowej w Warszawie oraz uczestniczy w projektach komercyjnych związanych z analizą danych dla firm z różnych sektorów. Jego główne zainteresowania związane są z teorią gier oraz analizą systemów złożonych.

\end{document}
