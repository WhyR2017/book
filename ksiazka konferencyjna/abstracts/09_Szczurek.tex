\providecommand{\main}{..} 
\documentclass[\main/boa.tex]{subfiles}

\begin{document}

\section{BioinfoRmatyka nowotworów}


\begin{minipage}{0.915\textwidth}
	\centering
  {\bf \LARGE \index[a]{Szczurek Ewa} Ewa Szczurek}
\end{minipage}



\begin{affiliations}
\begin{minipage}{0.915\textwidth}
\centering
\large Uniwersytet Warszawski  \\[1pt]
Kontakt: \href{mailto:szczurek@mimuw.edu.pl}{\nolinkurl{szczurek@mimuw.edu.pl}}\\
\end{minipage}
\end{affiliations}


Rak to choroba genomu. DNA komórek rakowych charakteryzują liczne alteracje, o przytłaczającej złożoności i różnrodności. W referacie przedstawię trzy podejścia analizy tak skomplikowanych danych genomicznych z komórek nowotworowych i wyciągania z nich konkretnych wniosków o mechanizmach tej choroby. Wszystkie trzy: muex, SurvLRT i lem, zostały zaimplementowane w R. 

\bio
Ewa Szczurek jest adiunktem w Instytucie Informatyki na wydziale Matematyki, Informatyki I Mechaniki Uniwersytetu Warszawskiego. Posiada dwa tytuły magistra informatyki: Uniwersytetu w Uppsali, w Szwecji, i Uniwersytetu Warszawskiego. Studia doktoranckie w dziedzinie bioinformatyki ukończyła w Instytucie Maxa Plancka w Berlinie. Odbyła dwa staże podoktorskie, jeden w Berlinie, a drugi na ETH w Zurychu. W swej pracy dydaktycznej stara się zainteresować statystyką studentów kierunku Bioinformatyka.

Od sześciu lat zajmuje się tematyką biologii obliczeniowej nowotworów. Obecnie naukowo rozgryza tematykę powstawania przerzutów w raku, a także koleżeńskie stosunki genów aktywnych w tej chorobie. Wszystkie tworzone w tej dziedzinie modele programuje w R.

\end{document}
