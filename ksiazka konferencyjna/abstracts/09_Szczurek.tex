\providecommand{\main}{..} 
\documentclass[\main/boa.tex]{subfiles}

\begin{document}

\section{BioinfoRmatyka nowotworów}


\begin{minipage}{0.915\textwidth}
	\centering
  {\bf \huge \index[a]{Szczurek Ewa} Ewa Szczurek}
\end{minipage}


\vskip 0.3cm

\begin{affiliations}
\begin{minipage}{0.915\textwidth}
\centering
\large Uniwersytet Warszawski  \\[5pt]
Kontakt: \href{mailto:szczurek@mimuw.edu.pl}{\nolinkurl{szczurek@mimuw.edu.pl}}\\
\end{minipage}
\end{affiliations}

\vskip 0.8cm

Rak to choroba genomu. DNA komórek rakowych charakteryzują liczne alteracje, o przytłaczającej złożoności i różnrodności. W referacie przedstawię trzy podejścia analizy tak skomplikowanych danych genomicznych z komórek nowotworowych i wyciągania z nich konkretnych wniosków o mechanizmach tej choroby. Wszystkie trzy: muex, SurvLRT i lem, zostały zaimplementowane w R. 

\end{document}
