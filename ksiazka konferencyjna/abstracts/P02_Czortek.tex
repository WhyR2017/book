\providecommand{\main}{..} 
\documentclass[\main/boa.tex]{subfiles}

\begin{document}

\section[Competition and tourism drive trade-off between vegetative and generative \\ reproduction of rare mountain species Carex lachenalii Schkuhr]{Competition and tourism drive trade-off between \\ vegetative and generative reproduction of rare \\ mountain species Carex lachenalii Schkuhr}

\begin{minipage}{0.915\textwidth}
	\centering
  {\bf \index[a]{Czortek Patryk} Patryk Czortek}
\end{minipage}


\begin{affiliations}
\begin{minipage}{0.915\textwidth}
\centering
Uniwersytet Warszawski  \\[-2pt]
\end{minipage}
\end{affiliations}

\vskip 0.3cm

 A result of sexual reproduction is long-distance spread at a meta-population level, whereas vegetative propagation contributes to increase population growth at a local scale. Trade-offs between these two components of reproduction reflect adaptation to the environment and may be a suitable way to understand threats of rare key-species. Example of such plant may be Carex lachenalii Schkuhr – a small tufted perennial sedge, occurring in extreme-specialized snowbed and acidophilous grasslands vegetation. This arctic-alpine species in the Tatra Mts occurs only in a few isolated sites. In 2016 we examined 96 localities of C. lachenalii. Maximum height and diameter, number of vegetative and generative stems, all vascular plants within 100m2 plots and the distance from the nearest trail was recorded for each sedge tuft, with the aim of determining whether tourism affects trade-off. To determine the role of competition and habitat filtering in shaping species composition of plant communities we calculated components of functional diversity using FD package. We used principal components analysis (PCA) for detecting relationships between species composition and populational traits of C. lachenalii, using vegan::envfit() function. For evaluation the impact of vegetation traits on trade-off we used generalized additive models (GAM) and chose the best model, based on Akaike’s Information Criterion (AIC). We found habitat-dependent relationships between all vegetation traits influencing the studied trade-off. Distance from the nearest trail did not influence the studied traits. 

\end{document}
