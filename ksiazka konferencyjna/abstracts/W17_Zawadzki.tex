\providecommand{\main}{..} 
\documentclass[\main/boa.tex]{subfiles}

\begin{document}

\section{Kiedy brakuje wydajności... R i C++ = Rcpp}

\begin{minipage}{0.915\textwidth}
\centering
{\bf \index[a]{Zawadzki Zygmunt} Zygmunt Zawadzki}
\end{minipage}

\vskip 0.3cm

\begin{affiliations}
\begin{minipage}{0.915\textwidth}
\centering
\large zstat  \\[2pt]
\end{minipage}
\end{affiliations}

\vskip 0.8cm

\opiswarsztatu Celem warsztatu jest nauczenie użytkowników wykorzystania pakietu Rcpp pozwalającego wykorzystać kod C++ w R w celu przyspieszenia krytycznych fragmentów obliczeń.

Uczestnik po skończonym warsztacie będzie potrafił:
\begin{enumerate}
\item przedstawić różnice w modelu zarządzania pamięcią w R i C++ i omówić konsekwencje które się z tym wiążą.
\item stworzyć prosty pakiet R wykorzystujący kod C++.
\item wykorzystać pakiet profvis do wyszukania najbardziej gorącego fragmentu kodu, który potencjalnie mógłby zostać przepisany z wykorzystaniem Rcpp.
\end{enumerate}

\planwarsztatu 
Wprowadzenie do Rcpp: Część I:
\begin{enumerate}
\item Kompilacja pierwszej funkcji wykorzystującej C++ w R.
\item Omówienie różnic pomiędzy językiem interpretowanym i kompilowanym na przykładzie R i Rcpp.
\item Szczegółowe omówienie struktur R dostępnych w C++ (NumericVector i NumericMatrix)
\item Przedstawienie STL - standardowej biblioteki szablonów jako dodatkowych struktur danych gotowych do wykorzystania.
\item Praktyczne prezentacja modeli zarządzania pamięcią w C++ - stworzenie kilku mini-funkcji prezentujących możliwe konsekwencje błędnej interakcji R i C++.
\end{enumerate}	 
Część II:
\begin{enumerate}
\item Profilowanie kodu z wykorzystaniem Rprof.
\item Wprowadzenie biblioteki RcppArmadillo do obliczeń macierzowych w C++.
\item Stworzenie prostego samplera Gibbsa wykorzystującego RcppArmadillo i funkcje R dostępne po stronie C++.
\end{enumerate}

\pakiety Rcpp, RcppArmadillo, profvis

\umiejetnosci Podstawy programowania w R:
\begin{enumerate}
	\item operacje na macierzach.
	\item pętle for.
\end{enumerate}
Znajomość C++ nie jest wymagana. Wszystkie potrzebne informacje dotyczące tego języka zostaną omówione w trakcie warsztatów.

\wymagania W przypadku systemu Windows pobranie i instalacja: Rtools - najnowsza wersja (https://cran.r-project.org/bin/windows/Rtools/)

Linux: wszystko powinno być zainstalowane (potrzebny jest kompilator gcc z obsługą standardu C++11 - jednak wszystkie w miarę nowe wersje powinny go mieć).

Mac: zainstalowane XCode.


\end{document}