\providecommand{\main}{..} 
\documentclass[\main/boa.tex]{subfiles}

\begin{document}

\subsection{Wyzwania stawiane przez technologie open source w biznesie}

\begin{minipage}{0.915\textwidth}
	\centering
  {\bf \index[a]{Olszewski Mikołaj} Mikołaj Olszewski}
    {\bf \index[a]{Bogucki Mikołaj} Mikołaj Bogucki }
\end{minipage}



\begin{affiliations}
\begin{minipage}{0.915\textwidth}
\centering
Pearson \\[-2pt]
\end{minipage}
\end{affiliations}

\vskip 0.3cm

W świecie współczesnej analityki danych coraz więcej firm rezygnuje z komercyjnych narzędzi analitycznych na rzecz oprogramowania open source, czego R jest świetnym przykładem. Prowadzi to nie tylko do redukcji kosztów ale często do rozwoju samej technologii przez firmę, która ją zaadoptowała.

Oprogramowanie open source takie jak R ma jednak pewne wady, np. pakiety nie działają zgodnie z oczekiwaniami, nowe wersje pakietów zmieniają lub usuwają stare funkcje, pakiety które zespół używa w codziennej pracy zostają całkowicie porzucone, zmuszając zespół do przyjęcia innych rozwiązań.

Jako analitycy danych w Pearsonie, wykorzystujemy R i Shiny jako główne narzędzia do przetwarzania danych, wizualizacji i raportowania. W naszej prezentacji przedstawimy faktyczne wyzwania, z którymi przyszło nam się zmierzyć w ciągu ostatnich kilku lat. Przedstawimy rozwiązania, które zastosowaliśmy oraz ich wpływ zarówno na bieżące projekty, jaki i na podejście zespołu do nowych problemów.

\end{document}
