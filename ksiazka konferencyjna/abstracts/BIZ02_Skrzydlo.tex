\providecommand{\main}{..} 
\documentclass[\main/boa.tex]{subfiles}

\begin{document}

\subsection{Kiedy data.frame pożera Workfile, czyli o tym jak przeprowadzamy stadko ekonometryków z klikania w EViewsie do pisania w R}

\begin{minipage}{0.915\textwidth}
	\centering
  {\bf \index[a]{Skrzydło Anna}Anna Skrzydło}
\end{minipage}


\begin{affiliations}
\begin{minipage}{0.915\textwidth}
\centering
MediaCom Warszawa Sp. z o. o. \\[-2pt]
\end{minipage}
\end{affiliations}

\vskip 0.3cm

Zmiana zawsze jest trudna. Każda zmiana. A szczególnie taka, która wymaga przeniesienia się z przyjaznego świata klikalnego oprogramowania do surowego, białego ekranu R Studio. Czy można zamienić kilkunastoosobowy zespół ekonometryków w programistów? Czy może to zajęcie tylko dla wybranych, którzy przywdziewając flanelowe koszule tworzą narzędzia jak najbardziej podobne do znanych i lubianych softów? Krótka opowieść o tym jak przenosimy nasz proces modelowania ekonometrycznego z EViewsa do R, jakie wyzwania stoją na naszej drodze i co dzięki tej zmianie zyskujemy. 

\end{document}
