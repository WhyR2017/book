\providecommand{\main}{..} 
\documentclass[\main/boa.tex]{subfiles}

\begin{document}

\section{Metody wizualizacji danych jakościowych w programie R}


\begin{minipage}{0.915\textwidth}
	\centering
  {\bf \huge \index[a]{Brzezińska Justyna} Justyna Brzezińska}
\end{minipage}


\vskip 0.3cm

\begin{affiliations}
\begin{minipage}{0.915\textwidth}
\centering
\large Uniwersytet Ekonomiczny w Katowicach   \\[5pt]
Kontakt: \href{mailto:justyna_brzezinska@ue.katowice.pl}{\nolinkurl{justyna_brzezinska@ue.katowice.pl}}\\
\end{minipage}
\end{affiliations}

\vskip 0.8cm

W referacie zaprezentowane zostaną metody wizualizacji danych jakościowych z użyciem odpowiednich pakietów programu R. Przedstawione zostaną podstawowe metody analizy danych jakościowych zapisanych w postaci dwu- i wielowymiarowych tablic kontyngencji, modele przeznaczone do analizy liczebności w tablicach kontyngencji, a także nowoczesne metody i techniki wizualizacji danych o charakterze niemetrycznym. W referacie przedstawione zostaną takie wykresy jak: wykres mozaikowy, sitkowy, asocjacji, dwuwarstwowy, czteropolowy, czy też cpcp oraz rmp. 

\end{document}
