\providecommand{\main}{..} 
\documentclass[\main/boa.tex]{subfiles}

\begin{document}

\section{Metody wizualizacji danych jakościowych w programie R}


\begin{minipage}{0.915\textwidth}
	\centering
  {\bf \huge \index[a]{Brzezińska Justyna} Justyna Brzezińska}
\end{minipage}


\vskip 0.3cm

\begin{affiliations}
\begin{minipage}{0.915\textwidth}
\centering
\large Uniwersytet Ekonomiczny w Katowicach   \\[5pt]
Kontakt: \href{mailto:justyna_brzezinska@ue.katowice.pl}{\nolinkurl{justyna_brzezinska@ue.katowice.pl}}\\
\end{minipage}
\end{affiliations}

\vskip 0.8cm

W referacie zaprezentowane zostaną metody wizualizacji danych jakościowych z użyciem odpowiednich pakietów programu R. Przedstawione zostaną podstawowe metody analizy danych jakościowych zapisanych w postaci dwu- i wielowymiarowych tablic kontyngencji, modele przeznaczone do analizy liczebności w tablicach kontyngencji, a także nowoczesne metody i techniki wizualizacji danych o charakterze niemetrycznym. W referacie przedstawione zostaną takie wykresy jak: wykres mozaikowy, sitkowy, asocjacji, dwuwarstwowy, czteropolowy, czy też cpcp oraz rmp. 

\bio
Justyna Brzezińska urodziła się w 1981 roku w Sosnowcu. Ukończyła szkołę podstawową w Sosnowcu oraz z wyróżnieniem IV Liceum Ogólnokształcące im. Stanisława Staszica w Sosnowcu. W 2000 roku rozpoczęła studia na Uniwersytecie Ekonomicznym w Katowicach na specjalności Statystyka i Ekonometria na Wydziale Zarządzania. W trakcie studiów była stypendystką programu Sokrates-Erasmus realizując semestr studiów na Uniwersytecie Aveiro w Portugalii. W 2005 roku obroniła pracę magisterską z zakresu skalowania wielowymiarowego uzyskując tytuł zawodowy magistra.

Po studiach odbyła liczne staże zagraniczne m. in. w Belgii, Malezji oraz Brazylii. W latach 2008-2012 była uczestnikiem Studiów Doktoranckich na Uniwersytecie Ekonomicznym w Katowicach. W 2012 roku wzięła udział w szkoleniach organizowanych w ramach programu LLP- Erasmus, które odbyła na Uniwersytecie Technicznym w Wilnie i Europa-Universität Viadrina we Frankfurcie nad Odrą. W 2014 roku obroniła prace doktorską pt.: „Modele logarytmiczno-liniowe i ich zastosowanie analizie zjawisk ekonomicznych” pod kierunkiem prof. dr hab. Eugeniusza Gatnara na Wydziale Zarządzania Uniwersytetu Ekonomicznego w Katowicach uzyskując stopień doktora nauk ekonomicznych w dyscyplinie ekonomia. Rozprawa ta uzyskała nagrodę Rektora Uniwersytetu Ekonomicznego w Katowicach.

W latach 2008-2012 pełniła funkcję sekretarza naukowego dziekana Wydziału Zarządzania Uniwersytetu Ekonomicznego w Katowicach. Od 2012 roku jest asystentem w Katedrze Analiz Gospodarczych i Finansowych na Wydziale Finansów i Ubezpieczeń Uniwersytetu Ekonomicznego w Katowicach. Kierowała grantem Narodowego Centrum Nauki pn. Modele logarytmiczno–liniowe w analizie danych jakościowych, prowadziła szkolenia z zakresu statystyki, a także gościnne wykłady naukowe z zakresu statystyki wielowymiarowej w uczelniach zagranicznych (Università degli Studi di Perugia, Università di Bologna, Università degli Studi di Milano). W 2015 była uczestnikiem szkolenia Tranmsformation.doc w University of Alberta w Kanadzie zorganizowanego przez Ministerstwo Nauki i Szkolnictwa Wyższego.

Jest autorką ponad trzydziestu publikacji naukowych, głównie w języku angielskim w prestiżowych czasopismach zagranicznych i krajowych, a także jednej monografii naukowej pt.: „Analiza logarytmiczno-liniowa. Teoria i zastosowania z wykorzystaniem programu R”. Jej artykuły były trzykrotnie nagradzane na konferencjach międzynarodowych. Jej zainteresowania naukowe obejmują: statystyczną analizę wielowymiarową, analizę danych jakościowych w szczególności modele logarytmiczno-liniowe, analizę klas ukrytych oraz modele teorii odpowiedzi na pozycje testowe (modele IRT).

\end{document}
