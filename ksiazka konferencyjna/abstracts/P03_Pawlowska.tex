\providecommand{\main}{..} 
\documentclass[\main/boa.tex]{subfiles}

\begin{document}

\section[Informacja publiczna trochę bardziej publiczna - dane z liczników rowerów \\ w Warszawie w Shiny]{Informacja publiczna trochę bardziej publiczna - dane z liczników rowerów w Warszawie w Shiny}

\begin{minipage}{0.915\textwidth}
	\centering
  {\bf \index[a]{Pawłowska Monika}  Monika Pawłowska}
\end{minipage}


\begin{affiliations}
\begin{minipage}{0.915\textwidth}
\centering
Instytut Biologii Doświadczalnej im. M. Nenckiego PAN   \\[-2pt]
\end{minipage}
\end{affiliations}

\vskip 0.3cm

 Dzisiejsze miasta zbierają ogromne ilości danych. Większość z nich stanowi informację publiczną, która powinna być udostępniana obywatelkom i obywatelom. Jednak aktywiści i urzędnicy zwykle nie mają ani odpowiednich umiejętności, aby narzędzi, żeby móc z tych danych skorzystać. Z kolei gotowe raporty publikowane są rzadko i odpowiadają tylko na wybrane pytania.
 
 Pochylając się nad tym problem, wzięliśmy pod lupę dane z automatycznych liczników, które od kilku lat zliczają ruch rowerowy w kilkunastu miejscach w Warszawie. Przy użyciu pakietu Shiny przygotowaliśmy aplikację dostępną pod adresem: \break http://greenelephant.pl/rowery. Pokazuje ona między innymi natężenie ruchu rowerowego w poszczególnych lokalizacjach w różnym czasie. Można z niej odczytać zależność liczby rowerów od temperatury powietrza i opadów. Dostępna też jest interaktywna mapa lokalizacji liczników.
 
 Shiny pozwala na przedstawienie danych w sposób przystępny, i elastyczny. Umożliwia ekspolarcję danych przez użytkowników bez wiedzy z dziedziny programowania i statystyki. Natomiast dzięki otwartemu kodowi aplikacji może być użyta jako przykład i łatwo zmodyfikowana na potrzeby wizualizacji danych z innych miast lub odmiennego charakteru.

\end{document}
