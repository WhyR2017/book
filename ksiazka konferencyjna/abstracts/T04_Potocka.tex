\providecommand{\main}{..} 
\documentclass[\main/boa.tex]{subfiles}

\begin{document}

\subsection{Blog z RMarkdownem i Jekyllem }

\begin{minipage}{0.915\textwidth}
	\centering
  {\bf \index[a]{Potocka Natalia} Natalia Potocka }
\end{minipage}

\vskip 0.3cm

\begin{affiliations}
\begin{minipage}{0.915\textwidth}
\centering
Grupa Wirtualna Polska \\[-2pt]
\end{minipage}
\end{affiliations}

\vskip 0.8cm

Dzięki pakietowi RMarkdown tworzenie stron internetowych jest naprawdę proste. W czasie prezentacji pokażę w jaki sposób pisać bloga lub prowadzić inną stronę internetową mając za narzędzie jedynie RStudio (i parę pakietów). Zaprezentuję proces tworzenia tego typu strony od A do Z oraz wskażę plusy i minusy takiego sposobu utrzymywania strony. 

\end{document}
