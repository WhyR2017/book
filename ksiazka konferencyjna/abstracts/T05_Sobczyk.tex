\providecommand{\main}{..} 
\documentclass[\main/boa.tex]{subfiles}

\begin{document}

\subsection{10 trików dla wizualizacji w ggplot2 }

\begin{minipage}{0.915\textwidth}
	\centering
  {\bf \index[a]{Sobczyk Piotr} Piotr Sobczyk}
\end{minipage}

\vskip 0.3cm

\begin{affiliations}
\begin{minipage}{0.915\textwidth}
\centering
Infermedica  \\[-2pt]
\end{minipage}
\end{affiliations}

\vskip 0.8cm

 Wszyscy chcemy tworzyć piękne wizualizacje i za pomocą R staje się to możliwe! Opowiem o niepodstawowych zastosowaniach ggplot2, który jest najbardziej popularnym pakietem do tworzenia grafiki w R. Jego zaletą jest to, że wystarczą jedynie dwie komendy aby stworzyć przyzwoicie wyglądający wykres. Co jednak gdy chcemy zrobić coś zaawansowanego? Podczas prezentacji przedstawię 10 trików na to jak ujarzmić ggplota. Wszystko na podstawie doświadczeń przy tworzeniu bloga szychtawdanych.pl 

\end{document}
