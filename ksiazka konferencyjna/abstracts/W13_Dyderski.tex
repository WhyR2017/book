\providecommand{\main}{..} 
\documentclass[\main/boa.tex]{subfiles}

\begin{document}

\section{Wróżenie z punktów - ordynacja w eksploracji danych}

\begin{minipage}{0.915\textwidth}
\centering
{\bf \index[a]{Dyderski Marcin K.} Marcin K. Dyderski}
\end{minipage}

\vskip 0.3cm

\begin{affiliations}
\begin{minipage}{0.915\textwidth}
\centering
\large Instytut Dendrologii Polskiej Akademii Nauk  \\[2pt]
\end{minipage}
\end{affiliations}

\vskip 0.8cm

\opiswarsztatu Warsztaty zakładają wprowadzenie do technik ordynacji - uporządkowania punktów w przestrzeni cech i redukcji wielowymiarowości do postaci zdatnej do wyrażenia za pomocą prostego wykresu. Ordynacja może być sama w sobie metodą do wykazania pewnych prawidłowości, może też jednak stanowić źródło do wyszukiwania zależności, które chcemy/musimy przedstawić później w bardziej wysublimowany sposób.

Celem warsztatów jest wskazanie możliwości zastosowania podstawowych technik ordynacyjnych do wyszukania zależności pomiędzy danymi. Planuję podczas warsztatów przeanalizować wraz z Uczestnikami trzy zbiory danych, w których będziemy szukać zależności związanych z problemem analitycznym. Oprócz tego chciałbym krótko omówić przykłady zastosowania tego typu analiz oraz najczęściej popełniane błędy.

Uczestnicy podczas warsztatów nauczą się:
\begin{enumerate}
\item jak przygotować dane do analiz z zastosowaniem ordynacji
\item w jaki sposób wykonać analizę głównych składowych (PCA), analizę korespondecji (CA) oraz kanoniczną analizę korespondencji (CCA)
\item w jakich warunkach dana analiza może być zastosowana, jakie ma wady oraz jakie ma ograniczenia
\item w jaki sposób interpretować uzyskane wyniki oraz jak przedstawić je graficznie w sposób przystępny i estetyczny
\end{enumerate}

\planwarsztatu
\begin{enumerate}
\item Czym jest ordynacja - wprowadzenie
\item Podział metod, zastosowania i przykłady
\item Przygotowanie danych, transformacje
\item Przypadek 1 - czym się różnią drzewa?
\item Przypadek 2 - co wpływa na naszą ocenę piwa?
\item Przypadek 3 - czym się różnią od siebie miasta?
\item Podsumowanie + informacje gdzie szukać dalej
\end{enumerate}	 

\pakiety vegan, ggplot2, gridExtra, scales

\umiejetnosci Podstawowa znajomość R: operacje na data.frame'ach, macierzach i listach, umiejętność tworzenia wykresów w pakiecie ggplot2

\wymagania komputer z R oraz zainstalowanymi pakietami

\end{document}
