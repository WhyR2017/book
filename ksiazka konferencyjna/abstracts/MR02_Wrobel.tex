\providecommand{\main}{..} 
\documentclass[\main/boa.tex]{subfiles}

\begin{document}

\subsection{Agregacja rozkładów przy pomocy kopul}

\begin{minipage}{0.915\textwidth}
	\centering
  {\bf \index[a]{Wróbel Adam} Adam Wróbel}
\end{minipage}


\begin{affiliations}
\begin{minipage}{0.915\textwidth}
\centering
UBS \\[-2pt]
\end{minipage}
\end{affiliations}

\vskip 0.3cm
Celem prezentacji jest zapoznanie uczestników z modelowaniem przy pomocy kopul poprzez przestawienie praktycznego zastosowania.

Sytuacja banku zależy od wielu czynników takich jak stopy procentowe, sytuacja na giełdzie, ceny na rynku nieruchomości. Każdy z tych czynników możemy modelować samodzielnie, ale w czasie kryzysu te czynniki mogą być ze sobą mocno skorelowane. Można to było zaobserwować w czasie ostatniego kryzysu, gdy krach na rynku nieruchomości wywołał kryzys na rynkach finansowych. Dlatego potrzebujemy struktury zależności miedzy czynnikami, aby mieć pełny obraz tego ile bank potrzebuje kapitału, aby przetrwać nawet w skrajnym scenariuszu. Taką strukturę zależności możemy zdefiniować wykorzystując kopule.

Pakiety: CDVine, ghyp, dplyr

\end{document}
