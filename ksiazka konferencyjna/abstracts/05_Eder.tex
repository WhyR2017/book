\providecommand{\main}{..} 
\documentclass[\main/boa.tex]{subfiles}

\begin{document}

\section{Kim naprawdę był Gall Anonim? \\ Zagadnienia statystycznej analizy tekstu}


\begin{minipage}{0.915\textwidth}
	\centering
  {\bf \huge \index[a]{Eder Maciej} Maciej Eder}
\end{minipage}


\vskip 0.3cm

\begin{affiliations}
\begin{minipage}{0.915\textwidth}
\centering
\large Instytut Języka Polskiego PAN  \\[5pt]
Kontakt: \href{mailto:maciejeder@gmail.com}{\nolinkurl{maciejeder@gmail.com}}\\
\end{minipage}
\end{affiliations}

\vskip 0.8cm

Wystąpienie będzie poświęcone analizie tekstu za pomocą kilku pakietów języka R, w tym atrybucji autorskiej opartej o statystyczne miary podobieństwa tesktów, a także szeroko rozumianej analizy stylu. Jako jeden z przykładów zostanie omówiony przykład autorstwa "Kroniki polskiej", przypisywanej tzw. Gallowi Anonimowi. W dalszej części wystąpienia zostanie przedstawiona metoda modelowania tematycznego (topic modeling) i jej zastosowania w analizie tekstu. 

\end{document}
