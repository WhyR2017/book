\providecommand{\main}{..} 
\documentclass[\main/boa.tex]{subfiles}

\begin{document}

\subsection{Jak wygrać więcej w lotto z R?}

\begin{minipage}{0.915\textwidth}
	\centering
  {\bf \index[a]{Kochański Błażej} Błażej Kochański}
\end{minipage}


\begin{affiliations}
\begin{minipage}{0.915\textwidth}
\centering
Politechnika Gdańska, Wydział Zarządzania i Ekonomii, Katedra Nauk Ekonomicznych  \\[-2pt]
\end{minipage}
\end{affiliations}

\vskip 0.3cm

 Dostępne w Internecie dane dotyczące poprzednich losowań oraz wygranych w loterii nazywanej kiedyś "Dużym Lotkiem" (6 z 49) mogą pomóc w ukształtowaniu optymalnej strategii gry. Jeden z kluczy: wybieranie mniej popularnych liczb. Celem jest maksymalizacja wartości oczekiwanej kwoty wygranej. Uzyskanie wartości oczekiwanej większej niż cena losu (R pomoże stwierdzić, czy to możliwe) prowadzi do pytania, jaką część majątku możemy inwestować? Z pomocą przychodzi tzw. kryterium Kelly'ego. 

\end{document}
