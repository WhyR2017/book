\providecommand{\main}{..} 
\documentclass[\main/boa.tex]{subfiles}

\begin{document}

\subsection{survminer - wykresy analizy przeżycia pełne informacji i elegancji}

\begin{minipage}{0.915\textwidth}
	\centering
  {\bf \index[a]{Kosiński Marcin} Marcin Kosiński}
\end{minipage}

\vskip 0.3cm

\begin{affiliations}
\begin{minipage}{0.915\textwidth}
\centering
Data Applications Designer \\[-2pt]
\end{minipage}
\end{affiliations}

\vskip 0.8cm

survminer to pakiet w R, który na scenie analizy przeżycia wypełnia lukę wizualizacji estymatorów krzywych przeżycia w duchu 'Grammar of Graphics' (ggplot2). W trakcie prezentacji przedstawię jak wyjątkowo elastyczne i konfigurowalne jest to narzędzie do tworzenia wykresów krzywych przeżycia. Wyjaśnię także czym są te wykresy oraz jak je interpretować. Warto rozumieć tę metodologię, ponieważ skala zastosowań analizy przeżycia jest rozpięta niemalże nad każdą dziedziną życia - od kontroli jakości żarówek, przez wyliczanie składek ubezpieczeniowych aż do badań klinicznych nad nowotworami. Jeżeli starczy czasu zaprezentuję także funkcjonalności survminer'a do diagnostyki i sprawdzenia założeń modelu Coxa proporcjonalnych hazardów - najbardziej popularnej metody statystycznej w analizie przeżycia, która niekoniecznie jest najlepiej rozumiana. 

\end{document}
