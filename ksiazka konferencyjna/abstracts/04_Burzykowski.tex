\providecommand{\main}{..} 
\documentclass[\main/boa.tex]{subfiles}

\begin{document}

\section{Dlaczego czasem R? Why (sometimes) R?}


\begin{minipage}{0.915\textwidth}
	\centering
  {\bf \huge \index[a]{Burzykowski Tomasz} Tomasz Burzykowski}
\end{minipage}


\vskip 0.3cm

\begin{affiliations}
\begin{minipage}{0.915\textwidth}
\centering
\large Hasselt University   \\[5pt]
Kontakt: \href{mailto:tomasz.burzykowski@uhasselt.be}{\nolinkurl{tomasz.burzykowski@uhasselt.be}}\\
\end{minipage}
\end{affiliations}

\vskip 0.8cm

W prezentacji przedstawiona zostanie odpowiedź na pytanie zawarte w tytule wystąpienia. Odpowiedź udzielona zostanie z punktu widzenia biostatystyka zajmującego się od ponad 30 lat analizą danych klinicznych, jak również organizacją prób klinicznych we współpracy z jednostkami akademickimi i firmami farmaceutycznymi. 

\bio
Tomasz Burzykowski uzyskał dyplom magistra w zastosowaniach matematyki (1990) na Uniwersytecie Warszawskim, a także dyplom magistra (1991) i doktora (2001) w biostatystyce na Uniwersytecie Hasselt (Belgia).

Pracuje jako profesor biostatystyki/bioinformatyki na Uniwersytecie Hasselt oraz jako wiceprezes ds. badań naukowych w International Drug Development Institute (Belgia).

Głównymi obszarami jego zainteresowań naukowych są metodologia prób klinicznych, meta-analizy prób klinicznych, walidacja zastępczych kryteriów oceny skuteczności leczenia, analiza przeżycia, oraz analiza danych “omicznych” (genomicznych itp.).

Jest współautorem, wraz z Andrzejem Gałeckim, książki

Linear Mixed-effects Models Using R. A Step-by-step Approach.

\end{document}
