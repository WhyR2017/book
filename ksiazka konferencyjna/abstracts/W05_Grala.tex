\providecommand{\main}{..} 
\documentclass[\main/boa.tex]{subfiles}

\begin{document}

\section{MicrosoftML - State of the art Machine Learning Microsoft}

\begin{minipage}{0.915\textwidth}
\centering
{\bf \index[a]{Grala Łukasz} Łukasz Grala}
\end{minipage}

\vskip 0.3cm

\begin{affiliations}
\begin{minipage}{0.915\textwidth}
\centering
\large Politechnika Poznańska Wydział Informatyki / TIDK - Data Scientist as a Services  \\[2pt]
\end{minipage}
\end{affiliations}

\vskip 0.8cm

\opiswarsztatu Firma Microsoft kupiła firmę Revolution i od tego momentu oferuje produkt R Server. Najnowsza odsłona tego produktu R Server 9.0 dostępna na różne platformy SQL Server (Windows i Linux), Hadoop, Teradata, Spark zawiera między innymi nową bibliotekę MicrosoftML. Biblioteka ta jest podsumowaniem pracy naukowej Microsoft Research w zakresie Machine Learningu. Są tam między innymi wydajne algorytmy do klasyfikacji, szukania anomalii, czy regresji. Dostępne są tam również algorytmy Deep Learning wykorzystujące GPU.

\planwarsztatu
\begin{enumerate}
\item Wprowadzenie do R Server
\item Algorytmy w MicrosoftML
\item Przykładowe scenariusze
\item Demonstracja Deep Learning z GPU
\end{enumerate}	 

\pakiety MicrosoftML (R Server - może być zainstalowany trial, lub wersja developer z SQL Server - Linux lub Windows)

\umiejetnosci Podstawy języka R, znajomość podstawowych klas problemów i algorytmów uczenia maszynowego

\wymagania Instalacja R Server 9. W razie wybrania mojego warsztatu przygotuje do tego stosowny manual.

\end{document}