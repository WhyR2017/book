\providecommand{\main}{..} 
\documentclass[\main/boa.tex]{subfiles}

\begin{document}

\section{Zawód rodzica a edukacja dziecka - wizualicja wyników badań PISA}

\begin{minipage}{0.915\textwidth}
	\centering
  {\bf \index[a]{Staniak Mateusz} Mateusz Staniak}
\end{minipage}

\vskip 0.3cm

\begin{affiliations}
\begin{minipage}{0.915\textwidth}
\centering
Uniwersytet Wrocławski \\[-2pt]
\end{minipage}
\end{affiliations}

\vskip 0.8cm

 Badanie PISA sprawdza wiedzę piętnastolatków z kilkudziesięciu krajów w dziedzinach czytania, matematyki i nauk przyrodnicznych. Na podstawie aplikacji napisanej pod opieką Przemysława Biecka, dostępnej pod adresem , pokażę, jak pakiety ggplot2 i shiny pozwalają odkryć zależności pomiędzy zawodami rodziców (które m.in. odzwierciedlają ich status społeczny) i wynikami ich dzieci oraz jak te zależności zmieniały się na przestrzeni lat 2006-2015.

\end{document}
