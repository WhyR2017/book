\providecommand{\main}{..} 
\documentclass[\main/boa.tex]{subfiles}

\begin{document}

\subsection{FSelectorRcpp – wolna od Java/Weka implementacja pakietu FSelector}

\begin{minipage}{0.915\textwidth}
	\centering
  {\bf \index[a]{Słomczyński Krzysztof} Krzysztof Słomczyński}
\end{minipage}

\vskip 0.3cm

\begin{affiliations}
\begin{minipage}{0.915\textwidth}
\centering
Appsilon \\[-2pt]
\end{minipage}
\end{affiliations}

\vskip 0.8cm

Celem prezentacji będzie zapoznanie użytkowników języka R z procesem powstawania oraz możliwościami pakietu FSelectorRcpp. Zostanie on też zestawiony z innymi popularnymi pakietami służącymi do selekcji zmiennych jak i z jego wcześniejszą – opartą na Java – implementacją. 

\end{document}
