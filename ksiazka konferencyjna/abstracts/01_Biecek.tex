\providecommand{\main}{..} 
\documentclass[\main/boa.tex]{subfiles}

\begin{document}

\section{Show Me Your Model}


\begin{minipage}{0.915\textwidth}
	\centering
  {\bf \huge \index[a]{Biecek Przemysław} Przemysław Biecek}
\end{minipage}


\vskip 0.3cm

\begin{affiliations}
\begin{minipage}{0.915\textwidth}
\centering
\large MI$^{2}$  \\[5pt]
Kontakt: \href{mailto:przemyslaw.biecek@gmail.com}{\nolinkurl{przemyslaw.biecek@gmail.com}}\\
\end{minipage}
\end{affiliations}

\vskip 0.8cm

Gramatyka grafiki (Wilkinson, Leland. 2006. The Grammar of Graphics) i jej implementacje (Wickham, Hadley. 2009. Ggplot2: Elegant Graphics for Data Analysis) zmieniły sposób w jaki myślimy o wizualizacji danych. Podobna rewolucja czeka wizualizacje modeli statystycznych. Podczas referatu przedstawię różne istniejące narzędzia do prezentacji modeli statystycznych (rms, forestmodel and regtools, survminer, ggRandomForests, factoextra, factorMerger) oraz zderzę je z jednolitym podejściem do przetwarzania modeli prezentowanym przez pakiet broom (Robinson, David. 2017. Broom: Convert Statistical Analysis Objects into Tidy Data Frames). Prezentacje zakończy zbiór doświadczeń dotyczących wizualizacji struktury modelu (Wickham, Hadley, Dianne Cook, and Heike Hofmann. 2015. Visualizing Statistical Models: Removing the Blindfold. Statistical Analysis). 

\end{document}
