\providecommand{\main}{..} 
\documentclass[\main/boa.tex]{subfiles}

\begin{document}

\subsection{Analiza sentymentu przy użyciu bibliotek Microsoft}

\begin{minipage}{0.915\textwidth}
	\centering
  {\bf \index[a]{Grala Łukasz} Łukasz Grala}
\end{minipage}

\vskip 0.3cm

\begin{affiliations}
\begin{minipage}{0.915\textwidth}
\centering
TIDK - Data Scientist as a Service  \\[-2pt]
\end{minipage}
\end{affiliations}

\vskip 0.8cm

Analiza sentymentu jest powszechnym wyzwaniem wielu dużych organizacji. Analizujemy treści poczty elektronicznej, dyskusji na forach, tekstów z komunikatorów, czy też różnorodnych portali społecznościowych. Analiz ta jest istotna z punktu widzenia budowania wizerunku firm, produktów, czy też osób. Algorytmów i metod jest wiele, w czasie sesji pokażemy jak to można wykonać dzięki gotowym biblioteką dostarczanym przez firmę Microsoft. Rozwiązanie dostępne zarówno w środowisku Windows, jak i Linux, z poziomu SQL Server, hurtowni danych Teradata, czy też rozwiązań pracujacych na HADOOP czy Spark



\end{document}
