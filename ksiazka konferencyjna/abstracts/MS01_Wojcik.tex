\providecommand{\main}{..} 
\documentclass[\main/boa.tex]{subfiles}

\begin{document}

\subsection{Modelowanie dynamiki rozkładu w R. Zastosowanie do analizy konwergencji na poziomie lokalnym}

\begin{minipage}{0.915\textwidth}
	\centering
  {\bf \index[a]{Wójcik Piotr} Piotr Wójcik}
\end{minipage}

\vskip 0.3cm

\begin{affiliations}
\begin{minipage}{0.915\textwidth}
\centering
Uniwersytet Warszawski, Wydział Nauk Ekonomicznych  \\[-2pt]
\end{minipage}
\end{affiliations}

\vskip 0.8cm

  W analizie różnych zjawisk społeczno-ekonomicznych (np. dochodu, osiągnięć edukacyjnych, stopy bezrobocia, preferencji politycznych, wielkości spożycia lodów itp.) często interesujące dla badacza jest ich zróżnicowanie w analizowanej próbie i zmiany tego zróżnicowania w czasie – patrz np. Magrini (2009). Najprostsze podejście ogranicza się do policzenia wybranej miary rozproszenia (np. współczynnika zmienności, współczynnika Giniego, Theila itp.) i porównania jego wartości w kolejnych okresach.
  
  Jednak pojedynczy wskaźnik nie mówi nic o zróżnicowaniu wewnątrz rozkładu. Dlatego inne popularne podejście bierze pod uwagę pełen rozkład badanego zjawiska i polega na porównywaniu histogramów albo jednowymiarowych estymatorów jądrowych w kolejnych okresach. To jednak wciąż nie mówi nic o mobilności wewnątrz rozkładu i nie pozwala formułować długookresowych przewidywań (rozkłady ergodyczne).
  
  Jest to możliwe kiedy dynamika rozkładu jest modelowana za pomocą macierzy przejścia (co wymaga dyskretyzacji rozkładu) albo estymatorów warunkowej funkcji gęstości po raz pierwszy zaproponowanych przez Quaha (1996). Celem prezentacji jest pokazanie jak różne podejścia do modelowania dynamiki rozkładu mogą być zastosowane w R, ze szczególnym uwzględnieniem macierzy przejścia i estymacji jądrowej. Zaprezentujemy zastosowanie w R metodologii umożliwiającej podsumowanie dwuwymiarowego warunkowego estymatora gęstości za pomocą jednowymiarowego rozkładu ergodycznego – patrz Gerolimetto i Magrini (2017).
  
  Przedstawimy także czytelne i atrakcyjne sposoby wizualizacji wyników estymacji. Praktyczne przykłady dotyczące modelowania procesów lokalnej konwergencji będą oparte na danych symulowanych oraz na rzeczywistych danych przestrzennych.
  
  Literatura Gerolimetto, Margherita, and Stefano Magrini. 2017. “A Novel Look at Long-Run Convergence Dynamics in the United States.” International Regional Science Review 40 (3). Magrini, Stefano. 2009. “Why Should We Analyse Convergence Through the Distribution Dynamics Approach?” Science Regionali 8: 5–34. Quah, Danny. 1996. “Twin Peaks: Growth and Convergence in Models Distribution Dynamics.” Economic Journal 106: 1045–55. Silverman, B.W. 1986. Density Estimation for Statistics and Data Analysis. Monographs on Statistics and Applied Probability. Londyn: Chapman; Hall.

\end{document}
